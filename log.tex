\documentclass{amsart}

\usepackage[colorlinks=true]{hyperref}
\usepackage{enumerate}
\usepackage{color}
\usepackage{mathrsfs}
\usepackage{tikz-cd}
\usepackage{amssymb}
\usepackage{centernot}
%\usepackage{spectralsequences} % currently an issue with some fragile macros that I've set

\usepackage{marginnote}
\renewcommand*{\marginfont}{\scriptsize\color{red}\sffamily}

\theoremstyle{plain}
\newtheorem{theorem}{Theorem}
\newtheorem{lemma}[theorem]{Lemma}
\newtheorem{proposition}[theorem]{Proposition}
\newtheorem{corollary}[theorem]{Corollary}

\theoremstyle{definition}
\newtheorem{definition}[theorem]{Definition}
\newtheorem{example}[theorem]{Example}
\newtheorem{exercise}[theorem]{Exercise}

\theoremstyle{remark}
\newtheorem{remark}[theorem]{Remark}

% Fonts
\newcommand{\A}{\mathbb{A}}
\newcommand{\C}{\mathbb{C}}
\newcommand{\F}{\mathbb{F}}
\newcommand{\R}{\mathbb{R}}
\newcommand{\Q}{\mathbb{Q}}
\newcommand{\Z}{\mathbb{Z}}
\newcommand{\N}{\mathbb{N}}
\newcommand{\G}{\mathbb{G}}
\newcommand{\fr}{\mathfrak}
%\newcommand{\sf}{\mathsf}

% Topology/geometry

\DeclareMathOperator{\Gr}{Gr}
\DeclareMathOperator{\Fl}{Fl}
\DeclareMathOperator{\PP}{\mathbb{P}}
\DeclareMathOperator{\Der}{Der}
\DeclareMathOperator{\Lie}{Lie}
\DeclareMathOperator{\SL}{SL}
\DeclareMathOperator{\GL}{GL}
\DeclareMathOperator{\SO}{SO}
\DeclareMathOperator{\UU}{U}
\DeclareMathOperator{\OO}{O}
\DeclareMathOperator{\Sp}{Sp}
\DeclareMathOperator{\HH}{H}
\DeclareMathOperator{\Symp}{Symp}
\DeclareMathOperator{\Spin}{Spin}
\DeclareMathOperator{\Pin}{Pin}
\DeclareMathOperator{\Td}{Td}
\DeclareMathOperator{\ind}{ind}
\DeclareMathOperator{\vect}{Vect}
\DeclareMathOperator{\Op}{Op}

% Representation theory

\DeclareMathOperator{\Ad}{Ad}
\DeclareMathOperator{\tr}{tr}
\DeclareMathOperator{\Str}{str}

% Algebra

\DeclareMathOperator{\End}{End}
\DeclareMathOperator{\Aut}{Aut}
\DeclareMathOperator{\Hom}{Hom}
\DeclareMathOperator{\sHom}{\mathscr{H}\!om}
\DeclareMathOperator{\sEnd}{\mathscr{E}\!nd}
\DeclareMathOperator{\id}{id}
\DeclareMathOperator{\irr}{irr}
\DeclareMathOperator{\Diff}{Diff}
\DeclareMathOperator{\gr}{gr}
\DeclareMathOperator{\im}{im}
\DeclareMathOperator{\ad}{ad}
\DeclareMathOperator{\rk}{rk}
\DeclareMathOperator{\Spec}{Spec}
\DeclareMathOperator{\Specm}{Specm}
\DeclareMathOperator{\Stab}{Stab}
\DeclareMathOperator{\Sym}{Sym}
\DeclareMathOperator{\Ext}{Ext}
\DeclareMathOperator{\ch}{ch}
\DeclareMathOperator{\cone}{cone}
\DeclareMathOperator{\cl}{Cl}

% Category theory

\DeclareMathOperator*{\colim}{colim}
\DeclareMathOperator*{\Map}{Map}

\makeatletter
\renewcommand\d[1]{\mspace{6mu}\mathrm{d}#1\@ifnextchar\d{\mspace{-3mu}}{}}
\makeatother

\newcommand{\naturalto}{%
    \mathrel{\vbox{\offinterlineskip
        \mathsurround=0pt
        \ialign{\hfil##\hfil\cr
            \normalfont\scalebox{1.2}{.}\cr
                            %      \noalign{\kern-.05ex}
        $\longrightarrow$\cr}
    }}%
}





\DeclareMathOperator{\DC}{DC} % differential characters
\DeclareMathOperator{\Del}{Del} % deligne cohomology
\newcommand{\fsl}[1]{{\centernot{#1}}}
\renewcommand\d{\mathsf{D}}

\title{Math log}
\author{Nilay Kumar}

\begin{document}
\maketitle
\tableofcontents

\section{June 6, 2017}

\subsection{Differential cohomology}
Cheeger and Simons \cite{cheeger-simons} define the group of differential characters
of degree $k$ (for $k>0$) as follows (let's shift the grading, following most other sources):
\begin{equation*}
    \DC^k(M; \Z) = \{h \in \Hom(Z_{k-1}(M; \Z), U(1)) \mid h\circ\partial \in \Omega^k(M) \}.
\end{equation*}
Here $Z_{k-1}(M)$ is the free abelian group of smooth singular cycles,
\marginnote{\cite{cheeger-simons} take normalized cubic chains. Is this necessary?}
and the condition
that $h\circ\partial \in \Omega^k(M)$ means that there exists an $\omega\in\Omega^k(M)$ such
that for any $c\in C_k(M)$,
\begin{equation*}
    h(\partial c) = \exp\left(2\pi i\int_c \omega\right) \in U(1),
\end{equation*}
i.e. the action of $h$ on boundaries is given by integration of some differential
form. \marginnote{What is integration against smooth singular chains?}
Notice that differential characters form an abelian group under pointwise multiplication
in the target $U(1)$.

Let's consider the simplest case where $k=1$. We claim that
\begin{equation*}
    \DC^1(M; \Z) \cong C^\infty(M, U(1))
\end{equation*}
Notice first that an integral 0-cycle of $M$ is just an integral-linear combination of points
in $M$, so $h$ is completely determined by its values on points of $M$. Hence we obtain
a function $\bar h:M\to U(1)$, which we wish to show is smooth. Fix some $p\in M$ -- let
us show that $\bar h$ is smooth at $p$. Let $U$ be a chart at $p$ and consider a 
ball $B\subset U$ centered at $p$. For each $q\in B$, consider the straight-line path
from $p$ to $q$ given $\gamma_q(t)=p + t(q-p)$. Since $\gamma_q\in C_1(M; \Z)$, we have
that
\begin{equation*}
    h(\partial\gamma_q) = h(\gamma_q(1)-\gamma_q(0)) = h(q - p) = h(q)h(p)^{-1}
\end{equation*}
but also that there exists $\omega\in \Omega^1(M)$ with (for all $q$)
\begin{equation*}
    h(\partial\gamma_q) = \exp\left( 2\pi i\int_{\gamma_q} \omega\right).
\end{equation*}
We conclude that for all $q\in B$
\begin{equation*}
    h(q) = h(p)\exp(2\pi i\int_{\gamma_q}\omega) =h(p)\exp(2\pi i\int_0^1\omega(\gamma_q(t)) dt),
\end{equation*}
which varies smooth with $q$.

Conversely suppose we are given a smooth function
$f:M\to U(1)$. This induces a map $h:Z_0(M;\Z)\to U(1)$ in the obvious way:
$h(\sum_ja_jp_j) = \sum_j a_j f(p_j)$. To construct the associated differential form $\omega$
choose, locally, a smooth real-valued lift $\bar f$ of $f$. Applying the exterior derivative
yields $\omega=d\bar f$ a global one-form on $X$.
Now given any $c\in C_1(M;\Z)$ we can write
$\partial c = \sum_ja_jp_j$ whence
\begin{equation*}
    h(\partial c) = \sum_j a_j f(p_j) = \int_{\partial c} \bar f \mod\Z= \int_c d\bar f\mod \Z,
\end{equation*}
so $h$ is a degree 1 differential character with associated 1-form $d\bar f$.
This completes the characterization of the first differential cohomology group.

One of the desirable features of differential characters is that the degree two group
classifies isomorphism classes of complex line bundles together with connection:
\begin{equation}
    \DC^2(M;\Z) \cong \{(L,\nabla)\to M\}.
    \label{classification}
\end{equation}
This is, in a sense, a geometric refinement of the cohomological classification
\marginnote{Write these two results out in some detail!}
\begin{equation*}
    \HH^2(M;\Z) \cong \{L\to M\}.
\end{equation*}
This latter classification follows from the representability of singular cohomology
by $[M, K(\Z, 2)]$ and of complex line bundles by $[M, BU(1)]$, and the fact that
\begin{equation*}
    K(\Z,2) \simeq BU(1) \simeq \C P^\infty.
\end{equation*}

There are various other definitions/viewpoints of differential characters. For convenience
and future reference I'll list them here:
\begin{enumerate}[(a)]
    \item there is an approach via smooth Deligne cohomology (c.f. \cite{brylinski} as well
        as Bunke's notes) that
        involves sheaves. Sean outlined to me some sort of modified \v Cech-de Rham complex
        that is likely coming from this Deligne story. I'd like to work that out soon.
        \marginnote{Work out this bicomplex that Sean mentioned.}
        I guess technically these aren't differential characters but an instance of differential
        cohomology -- see the mention of the axiomatic approach below.
    \item Hopkins and Singer in \cite{hopkins-singer}
        define the differential characters as the cohomology of a certain
        chain complex. I like this chain-level refinement because one asks the rather intuitive
        question: how can I combine of closed differential forms with integer periods
        (geometric data) with singular chains (topological data)? In particular, what is the
        (homotopy) pullback of the diagram
        \marginnote{What, roughly, is the relevance of closed differential forms with integral
        periods? This should be motivated from the Kostant/Souriau picture.}
        \begin{equation*}
            \begin{tikzcd}
                ?\rar\dar & \Omega^\bullet_\text{cl}(M)\dar\\
                C^\bullet(M;\Z)\rar & C^\bullet(M;\R)
            \end{tikzcd}
        \end{equation*}
        From the resulting chain complex(es), one can compute differential characters.
        Actually one can categorify slightly (not really adding much new information) and
        instead of working with the cohomology, work with certain categories of cochains.
        This allows us to access the groupoid of line bundles (with extra data) and not just
        the isomorphism classes. This is apparently important for geometric applications
        (involving, for instance, determinant lines).
        In fact, Hopkins and Singer go further by defining ``differential function spaces.''
        \marginnote{What, roughly, is HS's topological/physical reason for studying these objects?}
    \item Harvey and Lawson (c.f. \cite{hlz} and related papers)
        have an infinite-dimensional approach to defining differential characters. This seems rather
        complicated so perhaps (especially since I'm not familiar with currents) I'll save this for
        later, unless it's more clearly related to determinant lines, etc.
    \item Simons and Sullivan demonstrate in \cite{simons-sullivan} that differential characters
        satisfy certain axiomatic properties analogous to the Eilenberg-Steenrod axioms. In
        this sense, differential characters are a particular construction of ``ordinary
        differential cohomology.''
    \item towards more extreme generality, Bunke and company in \cite{bnv} analyse sheaves
        on the category of manifolds valued in nice stable $\infty$-categories and show that
        the structures of differential cohomology arise naturally\ldots or at least that's
        what the MathSciNet reviewer says. I'll perhaps ask Elden about the philosophy here,
        though it certainly won't be concretely useful to me.
\end{enumerate}

There is of course some story about differential refinements of generalized cohomology
theories (I believe there is some of this in \cite{hopkins-singer}) but I'll look into this
once I'm more familiar with spectra and complex K-theory.

\subsection{Eilenberg-Maclane spaces}
This talk of line bundles and second cohomology groups leads to a natural question: what
topological/geometric data do third cohomology groups classify? Apparently these objects
are called gerbes or 2-bundles, etc. There are a number of things to understand here. Maybe
it's first a good idea to understand what $K(\Z, 3)$ is. Sean pointed me to a webpage by
John Baez, ``Classifying spaces made easy'', which describes a way to describe $K(\Z, 2)$
and $K(\Z, 3)$, at least, using Hilbert spaces. Also: what is the intuition behind gerbes
and where do they appear in physics/geometry?


\section{June 7, 2017}

\subsection{Hopkins-Singer}
Today Sam and I worked through the basic constructions of differential cochains from
\cite{hopkins-singer}. One defines the chain complex $\check C(q)^\bullet(M)$ to be
the homotopy pullback:
\begin{equation*}
    \begin{tikzcd}
        \check C(q)^\bullet(M) \rar\dar & \Omega^{\bullet\geq q}(M)\dar \\
        C^\bullet(M;\Z) \rar & C^\bullet(M;\R)
    \end{tikzcd}
\end{equation*}
Sam wasn't sure off the top of his head why the given explicit formulas for this homotopy
pullback are correct, but he thought they seemed very reasonable given his intuition
for homotopy pullbacks in the category of spaces. I'll leave this for later, seeing
as the explicit formulae seem to be reasonable as we'll see below.
\marginnote{Understand the homotopy pullback of chain complexes.}

We checked explicitly that the map taking cohomology classes of differential cocycles to
differential characters,
\begin{equation*}
    (c^q, h^{q-1}, \omega) \mapsto (h^{q-1}\text{ mod }\Z, \omega)
\end{equation*}
% TODO renewcommand the mod in macros.tex without the spacing
is injective and surjective. This required nothing more than tracing through the
definitions of the relevant objects, so I won't reproduce it here (for the record,
surjectivity is immediate via taking a lift and injectivity requires using a bit
carefully the differential in the homotopy pullback). We conclude that the
Hopkins-Singer construction of this pullback cochain complex does indeed compute
differential cohomology:
\begin{equation*}
    \HH^q(\check C(q)^\bullet(M)) \cong \DC^q(M;\Z).
\end{equation*}

Actually, it is worth noting that in Hopkins-Singer
they think of differential characters as pairs $(\chi,\omega)$ so we had to
check that given a differential character $\chi$ there is a unique corresponding
closed integral-period form $\omega$. Let's check uniqueness. Suppose we a differential
character $\chi$ of degree $k$ (i.e. it eats things in $Z_{k-1}(M;\Z)$) associated to
two $k$-forms $\omega,\eta$. Then for any chain $c\in C_k(M;\Z)$,
\begin{equation*}
    0 = \chi(\partial c) - \chi(\partial c) = \int_c \omega-\eta \mod\Z.
\end{equation*}
We now have a differential form $\omega-\eta$ whose integral over any chain
(with $\Z$ coefficients) is an integer.\footnote{Beware of the distinction between
chain and cycle here! Periods of a form are its integrals over cycles!}
But if we
shrink any chain smoothly smaller and smaller we see that the integral must be zero
(measure theoretically clear),
but since these integrals take discrete values they must all be zero. 
Now since the map 
\begin{equation*}
    \Omega^\bullet(M) \hookrightarrow C^\bullet(M;\R).
\end{equation*}
is injective, it follows that $\omega=\eta$.
\marginnote{Does this argument work if we replaced $\Z$ with any proper subring of $\R$?}
The injectivity of this map is more or less clear: the integral locally is just given
as the integral of a smooth function on $\R^k$ but a continuous function is zero if every
integral is zero. If it were positive at a point there'd be a small neighborhood where it
was positive which would force the integral over that neighborhood to be positive.
We conclude that $\omega$ is uniquely associated to $\chi$.

That $\omega$ has integral periods is easy to see. Then
since for any chain in $c\in C_k(M;\Z)$ we have that the integral of $\omega$ modulo $\Z$
along the chain is $\chi(\partial c)$ it follows that if the chain is in fact a cycle
then $\partial c=0$ whence $\omega$ has integral periods.
That $\omega$ is closed is now an immediate consequence of Stokes' theorem:
\begin{equation*}
    \int_c d\omega \mod \Z= \int_{\partial c}\omega \mod\Z= \chi(\partial^2 c)=0.
\end{equation*}
Hence the integral of $d\omega$ along every $(k+1)$-chain is an integer whence by the argument
above $d\omega=0$.

Now recall that differential
characters form an abelian group by multiplication in the target. We think of
this as the addition in differential cohomology (hence the reason people write
$\R/\Z$ instead of $U(1)$ I think\ldots) but the multiplication is much more complicated,
it seems. In particular, the injection of forms into real cochains is not a ring
homomorphism. At the level of cohomology it is (this is part of the de Rham theorem)
but at the level of forms and chains one needs to choose some explicit homotopies
to write down a product. Maybe I'll come back to this later.
\marginnote{Come back to the ring structure on differential cohomology.}

% TODO determinants and theta characteristics
% TODO integration over smooth singular cycles
% TODO Thom forms/classes and orientations

As a side remark: Sean asked the following question. A differential character
is a map from some free abelian group to $U(1)$. Do we get anything interesting by
applying the classifying space functor $B$ to the character? We get a line bundle on
$BZ_k(M;\Z)$\ldots

\section{June 8, 2017}

\subsection{Principal bundles}
Let's return to some basics about vector/principal bundles, connections, and holonomy.
I'm familiar with connections on vector bundles, but not so much with parallel transport
and holonomy in the non-Riemannian case (and also with the apparently equivalent story
for principal bundles). This should be fairly straightforward (but lengthy) for me now,
though I remember as an undergrad Kobayashi/Nomizu was too intense.

If $G$ is a Lie group then a principal $G$-bundle $\pi:P\to M$ is a manifold $P$
equipped with a free right $G$-action such that $\pi$ is the quotient map and
$\pi$ is moreover locally $G$-trivial. A map $P\to P'$ of principal bundles over $M$
is a $G$-equivariant map lifting the identity map on $M$. This defines the category
of principal $G$-bundles over $M$.

Let's list a few facts with brief explanations.
\begin{enumerate}[(a)]
    \item Every map of principal $G$-bundles over $M$ is an isomorphism, whence the category
        is a groupoid. This is easily seen by writing an inverse for a map locally, using
        the inverse on $G$, and then globalizing (the inverse agree on overlaps as
        the original map does).
    \item If $\pi$ admits a global section $s:M\to P$ then $P$ is trivial, $P=M\times G$.
        Indeed, the map $(x,g)\mapsto g\cdot s(x)$ provides the desired diffeomorphism.
    \item By local triviality we have an open cover $U_\alpha$ and local trivializations
        $\phi_\alpha:P|_{U_\alpha}\to U_\alpha\times G$. On overlaps $U_\alpha\cap U_\beta$
        we obtain $\tau_{\beta\alpha}=\phi_\beta\circ\phi_\alpha^{-1}:U_\alpha\cap U_\beta G$.
        Notice that the maps $\tau$ satisfy a coycle condition. Conversely, given an
        open cover an a coycle $\tau$ as above we principal $G$-bundle.
        \marginnote{How do we think of a principal bundle as some sort of \v Cech data?
        When are two chains equivalent, and can we construct a groupoid like we have of
        principal bundles?}
    \item Let's recall the notion of an associated bundle. Let $\rho:G\to\GL_k V$ be a
        rank $k$ representation of $G$. Then we quotient $P\times V$ by the equivalence
        relation $(pg, v)\sim (p,\rho(g)v)$. We write the resulting space as $P\times_G V$,
        which is a rank $k$ vector bundle over $M$. Conversely, given a vector bundle
        one obtains a principal bundle by taking the frame bundle.
    \item The $G$-action on $P$ induces an assignment to each $\xi\in\fr g$ a vertical
        vector field $\hat \xi$ on $P$.
\end{enumerate}

A connection on $P$ is a $G$-equivariant splitting $TP = T(P/M) \oplus H$, i.e. the data
of a horizontal distribution $H$ such that $H_pg=(R_g)_*H_p$. Given this data we obtain
a global Lie algebra valued one-form on $P$ as follows. Define $\omega\in\Omega^1(P; \fr g)$
such that $\omega(X)$, for a vector field $X$ on $P$, is the unique $\xi\in\fr g$
with $\hat\xi=X^\text{v}$ where $X^\text{v}$ is the vertical component of $X$ (here
uniqueness follows from the fact that the induced vector vertical vector field is
nonvanishing since the $G$-action is free).
Notice that $\omega(X)=0$ if and only if $X$ is horizontal.
The connection one-form $\omega$ satisfies the following properties:
\begin{enumerate}[(a)]
    \item identity: $\omega(\hat \xi) = \xi$;
    \item equivariance: $R_g^*\omega = \Ad(g^{-1})\omega$
\end{enumerate}
Conversely, any $\fr g$-valued one-form satisfying these properties yields a connection
on $P$, by taking $H=\ker\omega$. Notice that we can pullback a connection along an
isomorphism of principal bundles -- this gives us a notion of an isomorphism of
principal bundles with connection.


\section{June 12, 2017}

\subsection{Singular cohomology}
Suppose we wanted to prove that \v Cech cohomology with coefficients in the constant
sheaf $\underline A$ for $A$ an abelian group computes the singular cohomology with
coefficients in $A$. One approach that I was thinking of is given by 
the following sketch (let's assume our space is reasonable, maybe the homotopy type
of a CW complex). Given an open cover
$\mathcal{U}$ we can construct a simplicial space called the \v Cech nerve of $\mathcal{U}$,
which I will denote by $\mathcal{U}_\bullet$.
\marginnote{We can probably understand \v Cech techniques more generally/categorically\ldots}
The $n$-simplices are the $n$-fold intersections of $n+1$ open sets of the cover. The
face maps are given by removing an open from the intersection and the degeneracy maps
are given by duplicating an open. I would like to claim that
\begin{equation*}
    \HH^\bullet(|\mathcal{U}_\bullet|;A) \cong \check\HH^\bullet(\mathcal{U}, \underline{A}).
\end{equation*}
From here, probably we can take colimits over refinements to obtain the result, because
the geometric realization will become homotopy equivalent to $X$ (this is a nontrivial
result).
\marginnote{Look at Segal's paper or HTT where this might be proved.}
I'm not exactly sure how to prove this isomorphism, but Piotr gave the following idea:
consider the Moore complex associated to $\mathcal{U}_\bullet$. Dualizing and taking
cohomology, we probably obtain the left hand side, though I'm not sure how to prove this.
% https://math.stackexchange.com/questions/1241500/homology-of-a-simplicial-set
What is clearer, however, is that the dual of the Moore complex yields the \v Cech
complex for $\mathcal{U}$ with coefficients in the constant sheaf $\underline A$.
Actually, I see why it gives coefficients in the constant presheaf, but I'm not sure
how to deal with the cases of disconnected intersections\ldots Perhaps this is only
true after passing to cohomology and then colimits (intuitively I would take as an
analogy the case of manifolds: in the colimit the opens become contractible).
\marginnote{How do we deal with this issue?}


\section{June 13, 2017}

\subsection{Sheaves}
Recall that a presheaf $\mathcal{F}$ of objects in a category $\sf C$ on a space
$X$ is a contravariant functor
from the category of open sets and inclusions on $X$ to $\sf C$.
Notice that it does not necessarily follow that $\mathcal{F}$ takes the empty
set to a final object in $\sf C$ -- indeed, a final object need not exist in
$\sf C$!
A sheaf is a presheaf satisfying a ``descent'' condition. Let's restrict to sheaves
of sets (or abelian groups, etc.).
\marginnote{Where does the terminology ``descent'' come from?}
In particular, given an open set $U\subset X$ and an open cover $\{U_\alpha\}$ of $U$,
and sections $s_\alpha$ over $U_\alpha$ such that the restriction of $s_\alpha$ to
$U_\alpha\cap U_\beta$ equals the restriction of $s_\beta$ to $U_\alpha\cap U_\beta$,
there exists a unique $s$ over $U$ restricting to the $s_\alpha$.
More categorically, we ask that the diagram
\begin{equation*}
    \begin{tikzcd}
        \mathcal{F}(U) \rar & \prod_\alpha \mathcal{F}(U_\alpha)
        \rar[shift left=3] \rar[shift right=3] & \prod_{\beta,\gamma}\mathcal{F}(U_\beta\cap U_\gamma)
    \end{tikzcd}
\end{equation*}
be an equalizer diagram in $\sf Set$ for all $\beta,\gamma$. Here the first map
is induced by the universal property of products and by the restriction maps. The
two parallel maps are given similarly but passing through restriction to $U_\beta$
and $U_\gamma$, respectively. For this diagram to be an equalizer in the category
of sets says precisely a global section is exactly the data of local sections that
agree on overlaps. I think when one generalizes to sheaves valued in higher categories
this diagram continues to the right\ldots

Given any presheaf $\mathcal{F}$, there is a universal sheaf
$\mathcal{F}\to \mathcal{F}^+$ associated to it, which is the sheaf of continuous
sections of the corresponding ``\'espace \'etal\'e'' (if I remember correctly from
Hartshorne). In other words one considers sections $X\to\sqcup_{x\in X}\mathcal{F}_x$
of a ``bundle of stalks'' that are locally given as the germ of a section of $\mathcal{F}$.
It is easy to see that this is a sheaf and it turns out that it has the same stalks
as $\mathcal{F}$ and maps from $\mathcal{F}$ into any sheaf factor uniquely through
$\mathcal{F}^+$. This construction is important, as many operations on sheaves
yield only presheaves in general (such as image, tensor product, etc).


\section{June 14, 2017}

\subsection{Presheaves on/as manifolds}
I've started to work through the paper of Freed and Hopkins, which has a few
long and very readable expository sections detailing presheaves, simplicial sets,
and weak equivalences.
Their main goal is to construct a universal $G$-connection on a universal
principal $G$-bundle $E_\nabla G\to B_\nabla G$ and then to compute
the de Rham complex of $B_\nabla G$. In particular, they interpret the
differential forms on $B_\nabla G$ as the universal characteristic forms
for principal bundles with connection. Their constructions take place
in a slight generalization of the category of manifolds. Let's look at
how this works.

A presheaf of sets on the category of manifolds is a functor $\sf{Man}^\text{op}\to\sf Set$.
Given any manifold $X$ we can associate to it the presheaf
\begin{equation*}
    F_X:M\mapsto \textsf{Man}(M, X),
\end{equation*}
where the right side is a set.  In this way the category of manifolds is a
subcategory of the category of set-valued presheaves on manifolds. This map
of categories is in fact fully faithful -- this is the Yoneda lemma/embedding,
which tells us that
\begin{equation*}
    \textsf{Psh}(F_X, F) \cong F(X).
\end{equation*}
We thus think of presheaves as generalized manifolds. This raise the question:
what are some presheaves that don't come from manifolds?  Consider, for instance,
the functor $\Omega^q$, which assigns to a manifold its degree $q$ differential
forms. In the setting of presheaves we consider $\Omega^q$ as a generalized
manifold, independent of any given manifold. Notice that the Yoneda lemma tells
us that
\begin{equation*}
    \textsf{Psh}(F_X,\Omega^q) \cong \Omega^q(X).
\end{equation*}
In other words, any differential $q$-form on a manifold $M$ is obtained by
pulling back the identity natural transformation on $\Omega^q$ along a natural
transformation $F_X\to\Omega^q$. With this in mind we think of the set of
$q$-forms on a generalized manifold $F$ as the set of maps of presheaves
from $F$ to $\Omega^q$.
In fact we have a ``complex'' of generalized manifolds $\Omega^\bullet$ 
where the differential is given by the differential on each manifold.
Notice that the differential is given by postcomposition by the exterior
derivative.
Notice that since $\Omega^\bullet(M)$ is cdga for
any manifold $M$, the set $\textsf{Psh}(F,\Omega^\bullet)$ is also a cdga.

The example Freed and Hopkins give is that of the de Rham complex of the
generalized manifold $\Omega^1$, i.e. the cdga $\textsf{Psh}(\Omega^1, \Omega^\bullet)$.
In other words, the degree $q$ elements of the complex are natural assignments
of a $q$-form to a 1-form (on any manifold! hence the naturality). Thinking
about this naively, there is only one way to do this: if $q$ is even,
take the exterior differential of the 1-form and then take the $q/2$th
wedge power, and if $q$ is odd, do take the $(q/2-1)$th wedge power and
wedge with the 1-form. Naturality is clear from the fact that wedge products
commute with pullbacks.
\marginnote{Is there an elementary proof that this is the only assignment
that works? Freed and Hopkins didn't see one.}

All this is to get to the presheaf definition of $B_\nabla G$. The obvious
definition is to assign to a manifold the set of isomorphism classes of
principal $G$-bundles with connection. The problem is that the resulting
presheaf is not a sheaf (unlike the more straightforward examples like
differential forms, function spaces, etc.) as can be seen in the example
of the circle covered by the two stereographic opens.
\marginnote{Understand why any connection on a principal bundle over the
interval is isomorphic to the trivial connection.}

\section{June 15, 2017}

\subsection{Functional analysis review}
Recall that a Banach space is a complete normed vector space. A common
example is that of $C^k(M)$ for $M$ a compact Riemannian manifold. The
norm on this vector space is defined as follows: choose any $N$ smooth
vector fields $Z_i$ on $M$ that span $T_pM$ for each $p$ (notice
$N\geqslant\dim M$). Then define
\begin{equation*}
    \lVert f\rVert_{C^k} = \sum_{l\leqslant k}\lVert Z_{j_1}\cdots Z_{j_l}f\rVert_\text{sup}.
\end{equation*}
It is easy to check that this defines a norm, and it turns out that
it is in fact complete (due to uniform convergence of all derivatives).
Notice that if we had picked a different set of such vector fields,
we'd obtain a different norm. Apparently, however, the topology on the
resulting space is independent of this choice.
\marginnote{Understand these details better.}

Another standard class of Banach spaces is that of $L^p$-spaces. Let
$(X,\mu)$ be a measure space and $L^p(X,\mu)$, for $p\in[1,\infty)$,
to be the space of functions $f$ (defined only almost everywhere, so
really these are equivalence classes, in order to actually obtain a
norm) with
\begin{equation*}
    \left( \int_X |f(x)|^p d\mu(x) \right)^{1/p}<\infty.
\end{equation*}
It is nontrivial to verify that these are indeed Banach spaces. By
convention we define $L^\infty(X,\mu)$ to be the (equivalence classes of)
bounded measurable functions on $X$ with the sup norm (where we take
a function in the equivalence class with the smallest sup).

Suppose we have a Banach space $V$ and a closed linear subspace
$W\subset V$. Then $W$ is again a Banach space: the norm of $V$ yields
a norm on $W$, and the completeness of $V$ together with $W$ being
closed implies completeness of $W$. In this case the quotient $V/W$
is naturally a Banach space with the norm
\begin{equation*}
    \lVert [v]\rVert = \inf_v \lVert v\rVert,
\end{equation*}
where the infimum is taken over all lifts of the class $[v]$. One checks
that this defines a norm (it's useful to rewrite it as the infimum over
all $w\in W$ of $\lVert v-w\rVert$ for a fixed lift $v$). It takes a bit
more work to show that under this norm the quotient is complete.

Let's turn to Hilbert spaces, which are complete inner product spaces,
hence a special class of Banach spaces, as the square of the inner product
of an element with itself defines a norm. One particularly nice property
of Hilbert spaces is that they have orthogonal projections. In particular,
if $H$ is a Hilbert space and $K\subset H$ is a closed linear subspace then
for $x\in H$ there exists a unique point $P_Kx\in K$ closest to $x$. To
prove this we use the parallelogram law
\begin{equation*}
    \lVert u+v\rVert^2 + \lVert u-v\rVert^2 = 2\lVert u\rVert^2 + 2\lVert v\rVert^2,
\end{equation*}
proved by expanding the right-hand side, and the following lemma.
\begin{lemma}
    If $K\subset H$ is a closed convex set then for each $x\in H$
    there is a unique $z\in K$ such that
    \begin{equation*}
        \lVert x-z\rVert = \inf_{y\in K}\lVert x-y\rVert.
    \end{equation*}
\end{lemma}
\begin{proof}
    There is a sequence $y_n\in K$ such that
    $\lVert y_n-x\rVert\to d=\inf_{y\in K}\lVert x-y\rVert$, and it is
    enough to show that this sequence is Cauchy. Using the parallelogram
    identity with $u=y_m-x$ and $v=x-y_n$, we get
    \begin{equation*}
        \lVert y_m-y_n\rVert^2 = 2\lVert y_n-x\rVert^2 + 2\lVert y_m-x\rVert^2
        - 4\lVert x-(y_n+y_m)/2\rVert^2.
    \end{equation*}
    The convexity of $K$ implies that $(y_n+y_m)/2\in K$ whence the last
    term is greater than or equal to $4d^2.$ Taking $m$ and $n$ large
    we find that $\lim_{m,n\to\infty}\lVert y_m-y_n\rVert^2\leqslant0$,
    which implies convergence.
\end{proof}
Applying the lemma to our case of $K$ a closed subspace, we can write
\begin{equation*}
    x = P_Kx + (x-P_Kx).
\end{equation*}
Let us check that $P_Kx$ is orthogonal to $K$. We notice that for any
$v\in K$ the quantity $\Delta(t) = \lVert x-P_Kx-tv\rVert^2$ is
minimized at $t=0$ by definition of $P_Kx$. Hence $\Delta'(0)=0$,
which implies that the real part of $(x-P_Kx, v)$ is zero for all
$v\in K$. Replacing $v$ with $iv$ shows that the imaginary part is zero,
from which we obtain the result. Hence we obtain a splitting
\begin{equation*}
    H = K\oplus K^\perp.
\end{equation*}

An easy corollary of the existence of orthogonal projections is the
Riesz representation theorem.
\begin{proposition}
    If $\phi:H\to\C$ is a continuous linear map then there exists a
    unique $f\in H$ such that
    \begin{equation*}
        \phi(v) = (v, f)
    \end{equation*}
    for all $v\in H$.
\end{proposition}
\begin{proof}
    Let $K=\ker \phi$, which is a closed subspace of $H$, by continuity of $\phi$.
    If $K=H$ then take $v=0$.
    Otherwise, $K^\perp\neq0$ so we can choose a nonzero $x_0\in K^\perp$ such that
    $\phi(x_0)=1$. We claim that $K^\perp$ is one-dimensional. Given any $y\in K^\perp$
    clearly $\phi$ sends $y-\phi(y)x_0$ to zero. This implies that $y-\phi(y)x_0$
    is both in $K$ and $K^\perp$, whence zero. Hence $K^\perp$ is spanned by $x_0$.
    The result now follows by setting $f=x_0/(x_0,x_0)$.
\end{proof}
This yields a conjugate linear isomorphism from the dual of $H$ to $H$.

\subsection{Thom isomorphism}
Hopkins and Singer refine Thom classes to the differential cohomology case.
I am, incidentally familiar with (or at least worked through at one point) the
Thom form construction of Mathai and Quillen, but unfortunately I don't know
the big picture related to Thom forms, Thom isomorphism, orientation, etc.
Sam said at some point that pushforwards in differential cohomology might
be useful in his work (since he thinks a lot about surface bundles, etc.)
so I should try to get a sense of the story.

Actually, Yajit just spent a good hour or so explaining to me the proof of
the Thom isomorphism theorem, which was probably good practice for his qual.
The statement in Milnor-Stasheff is as follows (up to notation).
\begin{theorem}[Thom isomorphism]
    Let $\pi:E\to M$ be an oriented rank $k$ vector bundle over $M$. Then
    $\HH^i(E, E\setminus M;\Z)$ is zero for $i<k$ and there is a unique class
    $u\in\HH^k(E,E\setminus M;\Z)$ called the Thom class, whose restriction
    to each fiber $F$, in $\HH^k(F, F\setminus 0;\Z)$ is equal to the
    preferred (according to the orientation) generator. Moreover, taking the
    cup product with the Thom class yields an isomorphism
    \begin{equation*}
        -\smile u : \HH^n(E;\Z) \xrightarrow{\sim} \HH^{n+k}(E,E\setminus M;\Z)
    \end{equation*}
    for each $n$.
\end{theorem}
The proof that Yajit outlined to me was for more or less the same theorem, just
stated slightly differently (using Thom spaces), and consisted of proving the theorem first for
trivial bundles (using some sort of relative K\"unneth formula) and then
using Mayer-Vietoris and the five lemma to extend it to locally trivial vector
bundles in general.

\section{June 16, 2017}

\subsection{The spin group}
Mike and I spent some time reviewing the basic definitions of Clifford algebras.
Our goal was to get to the definition of the group Spin, but we didn't quite get
there. We got stuck on the following proposition. Let $V$ be a real vector
space with inner product $Q$.
\begin{proposition}
    The image in the Clifford algebra $C(V,Q)$ of the degree two differential
    forms is a Lie subalgebra isomorphic to $\fr{so}(V,Q)$.
\end{proposition}
\begin{proof}
    First one checks that the image $C^2(V)$ is closed under commutator.
    This follows from the fact that if $c_i$ is the image of an orthonormal
    basis of $V$ (with respect to $Q$) then
    \begin{equation*}
        c_ic_j + c_jc_i + 2\delta_{ij} = 0.
    \end{equation*}
    Now the map $C^2(V)\to\fr{so}(V,Q)$ is defined by sending $a\mapsto[a,-]$.
    Let us write $\tau(a)$ for the endomorphism sending $v\mapsto \tau(a)v$.
    A similar argument as above shows that $\tau(a)$ takes $C^1(V)$ to $C^1(V)$,
    and moreover $\tau(a)$ is in $\fr{so}(V,Q)$ because
    \begin{equation*}
        Q(\tau(a)v,w) + Q(v,\tau(a) w) = 0
    \end{equation*}
    using the Clifford relation together with the Jacobi identity (in particular
    the expression reduces to $-[a,[v,w]]/2$ but $[v,w]$ is by the Clifford relation
    a scalar). This map is an isomorphism because it is injective and because
    the source and target are vector spaces of the same dimension $n(n-1)/2$.
\end{proof}
There are two things we didn't understand: where did we use positivity as opposed
to just nondegeneracy of $Q$, and why is the map injective?
\marginnote{Understand this! It shouldn't be difficult.}
If we understand this result then the spin group $\Spin(V,Q)$ is obtained by
exponentiating the finite-dimensional Lie algebra $C^2(V)$ inside the Clifford
algebra $C(V)$. We were unclear exactly what this meant -- as a set one can
consider all symbols of the form $\exp(t a)$, but what is the Lie group structure
on this set?
\marginnote{Understand this! Find a reference or ask someone?}
One thing left to do is to understand the anti-automorphism on the Clifford
algebra (and self-adjointness in the context of Clifford modules) and how
this relates to the double covering of $\Spin(V,Q)\to \SO(V,Q)$.

\subsection{Pushforward bundle}
Aaron Naber was in the common room for a few minutes so I took the chance to ask
him this issue I've been confused about for a while (and that both Aaron and Steve
brought up during my qual) about the definition of $\pi_*E$. Recall the setup: we have a
smooth proper map $\pi:M\to B$ and a finite-rank vector bundle $E\to M$. We would
like to define a infinite-dimensional vector bundle $\pi_*E\to B$ whose fibers
are given, above $z$,
\begin{equation*}
    {\pi_*E}_z = \Gamma(\pi^{-1}(z), E|_{\pi^{-1}(z)}).
\end{equation*}
Of course we can define this object fiberwise\ldots but why is there a locally
trivial space with these fibers? Aaron suggested doing the following: choose a
small enough $U\subset B$ over which $M$ is trivialized, with model fiber $F$.
The fibers are identified, via this trivialization, with the model fiber $F$,
and this identification should yields identifications of the spaces of sections
of $E$ restricted to these fibers.

\subsection{Differential Thom cocycles}
Hopkins and Singer are concerned with families in the following sense: they
take $E\to S$ to be a smooth proper map of manifolds. In fact, this is not quite
true. They work with manifolds with corners in general, and so they work with
``neat'' maps, but I'll ignore this technicality for now. The motivation for
this Thom business is that we would like to formulate a notion of fiber integration
for (ordinary) differential cohomology.

\section{June 19, 2017}

\subsection{Thom forms}
Let $\pi:E\to M$ be an oriented vector bundle of rank $k$ over a compact oriented
manifold $M$.
\begin{definition}
    A Thom form for $E$ is a closed, compactly supported, differential form
    $U\in \mathcal{A}^k_c(E)$ such that
    \begin{equation*}
        (2\pi)^{-k/2}\int_{E/M} U = 1 \in \mathcal{A}^0(M).
    \end{equation*}
\end{definition}
The following Thom isomorphism theorem justifies the name ``Thom form.''
\begin{theorem}
    With $\pi$ and $U$ as above, the maps
    \begin{align*}
        \HH^\bullet_c(E) & \to \HH^{\bullet-k}(M)\\
        \beta \mapsto & (2\pi)^{-k/2}\pi_*\beta
    \end{align*}
    and
    \begin{align*}
        \HH^\bullet(M) &\to \HH^{\bullet+k}_c(E)\\
        \alpha \mapsto & U\wedge \pi^*\alpha
    \end{align*}
    are inverses.
\end{theorem}

This theorem is not particularly difficult to prove, but I'll come back to it
later.
\marginnote{Come back and prove this theorem!}
For now I want to work through the construction (due to Mathai and Quillen
in \cite{mq}, though I am following the exposition of \cite{bgv}) of the Thom form.
The Thom form is given by the formula
\begin{equation*}
    U = \varepsilon(k) Te^{-|x|^2/2+i\nabla x+F}.
\end{equation*}
which we now explain.

Given any oriented Riemannian vector bundle $E$ of rank $k$ over $M$, we have
the Berezin integral
\begin{equation*}
    T: \mathcal{A}(M, \Lambda E)\to \mathcal{A}(M)
\end{equation*}
which returns the coefficient of the top degree element of $\Lambda E$ that is
determined by the orientation. One has the following useful formula.
\begin{lemma}
    If $\nabla$ is a connection on $E$ compatible with the metric then for any
    $\alpha\in \mathcal{A}(M,\Lambda E)$ we have
    \begin{equation*}
        dT(\alpha) = T(\nabla \alpha).
    \end{equation*}
\end{lemma}
\begin{proof}
    Fix, locally, an oriented orthonormal frame $e^i$ for $E^*$. Then $T$ is
    given by pairing against $e^1\wedge\cdots\wedge e^k$. Using the definition
    of the dual connection,
    \begin{equation*}
        dT(\alpha) = (\nabla T)(\alpha) + T(\nabla \alpha).
    \end{equation*}
    The first term vanishes,
    \begin{equation*}
        \nabla T = \sum e^1\wedge\cdots \nabla e^i\wedge\cdots\wedge e^k = 0
    \end{equation*}
    since the connection is compatible with the metric (the connection one-form
    matrix is traceless, in particular antisymmetric).
\end{proof}

Consider now $\pi: E\to M$ as before, where we equip $E$ with a metric $(\cdot,\cdot)$
and compatible connection $\nabla^E$. We may pull back $E$ along $\pi$
\begin{equation*}
    \begin{tikzcd}
        \pi^*E \rar\dar & E\dar\\
        E\rar & M
    \end{tikzcd}
\end{equation*}
as well as the connection $\nabla^{\pi^*E}=\pi^*\nabla^E$.
We obtain a bigraded algebra
\begin{equation*}
    \mathcal{A}^{i,j} = \mathcal{A}^i(E, \Lambda^j(\pi^*E))
\end{equation*}
of multivector-valued forms on the left-hand side of the diagram. The connection
defines a linear map $\nabla: \mathcal{A}^{i,j}\to \mathcal{A}^{i+1,j}$.
We have a few distinguished elements of the algebra $\mathcal{A}$. The first
is $x\in \mathcal{A}^{0,1}$, which denotes the tautological section. This
is the smooth section assigning to a point $(p,v)$ the vector $v$ itself.
Using the metric and the connection we obtain elements $|x|^2\in \mathcal{A}^{0,0}$
and $\nabla x\in \mathcal{A}^{1,1}$.
Notice that we can identify antisymmetric matrices (elements of $\fr{so}(E)$)
with $\Lambda^2E$ under the map
\begin{equation*}
    A\in\fr{so}(V) \mapsto \sum_{i<j} (Ae_i, e_j)e_i\wedge e_j.
\end{equation*}
Here $e_i$ is a oriented orthonormal local frame for $E$.
\marginnote{Why is this assignment global?}
That this map is injective is clear.
Then the curvature $(\nabla^E)^2\in \mathcal{A}^2(M,\fr{so}(E))$
pulls back to an element, under this identification, $F\in \mathcal{A}^{2,2}$.

We will also need the contraction $a(s):\mathcal{A}^{i,j}\to \mathcal{A}^{i,j-1}$
for any $s\in \mathcal{A}^{0,1}$,
\begin{equation*}
    a(s)(\alpha\otimes(s_1\wedge\cdots\wedge s_j)) = \sum_{m=1}^j(-1)^{|\alpha|+m-1}(s,s_k)\alpha
    \otimes(s_1\wedge\cdots\wedge \hat s_k\wedge\cdots\wedge s_j)
\end{equation*}
for $\alpha\in \mathcal{A}(E)$ homogeneous and $s_k\in \mathcal{A}^{0,1}$.
Notice that for any $s\in \mathcal{A}^{0,1}$ and $\alpha\in \mathcal{A}$, we have
$T(a(s)\alpha)=0$, since $a(s)\alpha$ has no component in $\mathcal{A}^{\bullet, k}$.
Hence
\begin{equation*}
    dT(\alpha) = T( (\nabla + a(s))\alpha).
\end{equation*}

\begin{lemma}
    Write
    \begin{equation*}
        \Omega = \frac{|x|^2}{2} + i\nabla x + f \in \mathcal{A}.
    \end{equation*}
    Then
    \begin{equation*}
        (\nabla - ia(x))\Omega = 0.
    \end{equation*}
    Moreover, for any smooth $\rho\in C^\infty(\R)$, define $\rho(\Omega)\in \mathcal{A}$
    by the formula
    \begin{equation*}
        \rho(\Omega) = \sum_{m=0}^k \frac{\rho^{(m)}(|x|^2/2)}{m!}(i\nabla x+F)^m.
    \end{equation*}
    Then $(\nabla-ia(x))\rho(\Omega)=0$.
\end{lemma}
\begin{proof}
    First note that $a(x)x=|x|^2$ and $a(x)|x|^2=0$. Compatibility gives us that
    $\nabla |x|^2=2(\nabla x, x)=-2a(x)\nabla x$, where the second equality follows
    from the definition of $a$. The Bianchi identity yields $\nabla F=0$ and the
    definition of the curvature yields $\nabla(\nabla x)=a(x)F$. These computations
    together show that $(\nabla-ia(x))\Omega=0$. The rest of the proposition follows
    from the fact that $\nabla-ia(x)$ is a derivation of the algebra $\mathcal{A}$.
\end{proof}

Since $\Omega\in\sum_{0\leq m\leq 2}\mathcal{A}^{m,m}$ we see that
$\rho(\Omega)\in\sum_{0\leq m\leq k}\mathcal{A}^{k,k}.$ Thus $T\rho(\Omega)\in \mathcal{A}^k(E)$
and we have that
\begin{equation*}
    dT(\rho(\Omega)) = T( (\nabla-ia(x))\rho(\Omega)) = 0.
\end{equation*}
We conclude that $T(\rho(\Omega))$ is a closed $k$-form on $E$.
The Thom form is now defined by taking $\rho(x)=\varepsilon(k)e^{-x}$
where $\varepsilon(k)=1$ if $k$ is even and $i$ if $k$ is odd:
\begin{equation*}
    U = \varepsilon(k)Te^{-\Omega} = \varepsilon(k) Te^{-|x|^2+i\nabla x+F}.
\end{equation*}
Note that the Thom form depends on the orientation (through $T$), metric,
and connection.

\begin{proposition}
    The form $(2\pi)^{-k/2}U$ has integral 1 over each fiber.
\end{proposition}
\begin{proof}
    We compute by restricting to the fiber above a point $p\in M$, so we work
    on $\pi^*E_p\to E_p\to \{p\}$. Now $x$ is the identity map $V\to V$, thought
    of as an element $\mathcal{A}^0(V,V)$. The pullback connection on $\pi^*E_p$
    is now the trivial connection (since the base of $E_p\to\{p\}$ is a point),
    whence the Thom form over $p$ is given $\varepsilon(k)T\exp(-|x|^2/2-idx)$.
    Noting that $dx=\sum dx^i\otimes e_i$ once we have chosen an (oriented orthonormal)
    basis, we obtain the result.
    \marginnote{Does this really makes sense? Work through in detail!}
\end{proof}


\section{June 20, 2017}

I met with Sam today for a few hours and I sketched to him this Mathai-Quillen
construction of a Thom form. Then he described to me the Deligne cohomology
model and briefly mentioned another model that uses differential forms on
simplicial manifolds (this approach is apparently useful in the study of flat
surface bundles\ldots). We then briefly thought about the product structure
on ordinary differential cohomology but then got pretty confused.

There are a few basic ideas about differential cohomology I'd like to work through
in detail soon:
\begin{enumerate}[(a)]
    \item why does $\check\HH^2(M;\Z)$ classify line bundles with connection? What
        does the abelian group structure correspond to?
    \item work through the various exact sequences that show up
    \item what is the product on differential cohomology (in the 3 main models)?
    \item work through the construction of Deligne cohomology, and understand
        why it computes differential cohomology
    \item where and why does differential cohomology arise in the physics literature?
        What is the connection to index theory?
    \item what is the definition of differential $K$-theory?
    \item what are slant products? work through the Pontryagin-Thom construction
        and understand fiber integration
\end{enumerate}

\subsection{Differential cohomology}

There is a bit of weirdness going on with the gradings in differential cohomology.
I quite like the Hopkins-Singer cochain model, so let's work with that. I first
claim that if $M$ has $r$ connected components then
\begin{equation*}
    \DC^0(M;\Z) \cong \Z^r.
\end{equation*}
This follows immediately from the definition of the differential cochain complexes
$C(q)^n(M)$. In particular, recall that
\begin{equation*}
    C(q)^q(M) = C^q(M;\Z)\times C^{q-1}(M;\R)\times \Omega^q(M)
\end{equation*}
with differential given
\begin{equation*}
    d(c,h,\omega) = (\delta c, \omega - c-\delta h, d\omega).
\end{equation*}
For $q=0$ the second factor is zero whence the kernel of $d$ is precisely those triples
$(c,0,\omega)$ for which $\delta c = d\omega = 0$ and $\omega - c=0$. Here $\omega$ is
a closed zero-form on $M$, i.e. locally constant. The equation $\omega = c$ forces
$\omega$ and $c$ to be the same locally constant integer. This proves the claim.

We argued above that $\DC^1(M;\Z)$ is the group of $U(1)$-valued functions on $M$ and
at least motivated that $\DC^2(M;\Z)$ is the group of isomorphism classes of $U(1)$-bundles
with connection on $M$. If we were to give the graded abelian group $\DC^\bullet(M;\Z)$ the
structure of a ring, we would for instance have a method of taking two $U(1)$-valued functions
and producing a $U(1)$-bundle with connection. Similarly one could take a $U(1)$-valued function
and a $U(1)$-bundle with connection and produce a gerbe. What does this product look like?

It is a straightforward computation that the product given in Hopkins-Singer on cochains
descends to cohomology, assuming we believe the given statements about the homotopy between
the cup and wedge products. I can write this out tomorrow. My goal for tomorrow is
to work out some explicit examples of this product.

More of a stretch goal: understand roughly the picture of anomalies in quantum field theories
and the relation to differential cohomology and index theory (as well as this paper of Moore
and Lukic).

\section{June 21, 2017}

\subsection{Sheaves of spectra on manifolds}
I spent some time with Elden today skimming through \cite{bnv} to get a sense of where
differential cohomology comes from in the most abstract sense. He explained a lot of the
scary $\infty$-categorical terminology so I think I got a reasonable idea of what's going
on. Let me just try to sketch the main idea. Suppose we have a stable $\infty$-category
$\sf C$ and a $\sf C$-valued (homotopy) sheaf $\hat E$ on $\sf Mfld$ (here we take
the site with coverings being \'etale maps, and we work with manifolds with corners
into order to be able to use topological simplices). The authors show that there exist
endofunctors of $\sf C$-valued sheaves $\mathcal{H}$ and $\mathcal{Z}$ together with maps
\begin{equation*}
    \begin{tikzcd}
        \hat E \rar{R}\dar{I} & \mathcal{Z}(\hat E)\\
        \mathcal{H}(\hat E)
    \end{tikzcd}
\end{equation*}
which fit into a certain larger diagram involving various other constructions. One
defines $\mathcal{H}$ as an operator that takes a sheaf and ``homotopifies'' it
in the sense that its value on $M\times\Delta^1$ are equivalent to its value on $M$.
In the differential cohomology context $\mathcal{H}$ gets rid of the ``differential''
part of the differential cohomology theory, which is not homotopy invariant. Hence
we think of $I$ as the forgetful map assigning the underlying topological class.
On the other hand, $\mathcal{Z}$ (which is defined as the cofiber of a certain (co)unit
map in an adjunction) only keeps the geometric data. Thus we think of $R$ as the forgetful
map assigning the underlying differential form data. It's quite neat that such
structures arise in this generality and are pretty straightforwardly interpretable in
the usual case. Another nice result in this paper is a homotopy formula that
measures, in a sense, the failure of $\hat E$ to be homotopy invariant in a rather
intuitive way. I'm pleasantly surprised by how interesting this general story is, even though
it's useful probably only as philosophy when it comes to index theory and physics.
I think Kyle is also interested in reading this paper, so it'd be nice to learn the
relevant category theory and topology together as we work through this.

A rough outline of things to work through:
\begin{enumerate}[(a)]
    \item sites and Grothendieck topologies;
    \item basics of $\infty$- and stable $\infty$-categories;
    \item sheaves on manifolds
    \item \ldots
\end{enumerate}

\subsection{Motivation for Mathai-Quillen}

Sam asked me a good question yesterday: where does the formula for the Thom form given
by Mathai and Quillen come from? The way that the construction is presented in
\cite{bgv} is pretty much completely unmotivated, with only a passing remark about
Chern characters and family index.

The original paper of Mathai and Quillen is much more illuminating. In particular,
let's outline what they do in section 4 (they have an alternate construction later
that may be related to that of \cite{bgv}?)
\marginnote{What is this alternate construction?}

Based on skimming the paper here's my guess. Fix a vector bundle of even rank
$\pi:E\to M$ and suppose $E$ is equipped with a spin structure. 
They begin by stating in (4.1) some sort of cohomological index theorem:
if $i:M\to E$ is the zero section then
\begin{equation*}
    \ch(i_!1) = \pi^*(\hat A(E)^{-1})i_*1.
\end{equation*}
I don't know much about $K$-theory unforunately but $1$ is probably a trivial class
in the $K$-theory of the base and $i_*$ is some sort of wrong way map. On the right
we have the pullback of the inverse of the A-hat genus wedged with the Thom class
in cohomology $i_*1$. Mathai and Quillen cite Atiyah-Bott-Schapiro
in an explicit representation for the $i_!1$. This representation -- a vector bundle
over the total space $E$ -- is naturally equipped with a Dirac operator $\sf D$ and
nn odd endomorphism $L$. This yields a superconnection $\mathsf{D}+L$ on the representative,
whence the left hand-side of the formula above can be computed as usual, as a supertrace
of an exponential of the curvature of the superconnection. Carrying out this computation
we obtain the inverse $\hat A$-genus of $E$ times a certain form. This form thus serves
as a representative for the Thom class. This construction is where the exponential form
originates from.


\subsection{Products in differential cohomology}
Let's briefly outline the product in differential cohomology. Obviously the products
at the level of cochains will be quite different, but we expect them to be
graded-commutative.
\subsubsection{Hopkins-Singer}
Recall that the Hopkins-Singer model for differential cochains was a certain
homotopy pullback in the (model, $\infty$-, etc.) category of cochain complexes.
Cochains are triples $(c, h, \omega)$ where the first component is a integral
cochain, the second component is a real cochain of one degree lower, and $\omega$
is a differential form. Apparently the fact that the wedge product of forms
is not to sent to the cup products on cochains by the integration map complicates
matters.
\marginnote{What exactly is the complication? Presumably it appears in the descent to
cohomology?}
We first are asked to fix a chain homotopy between the cup and wedge products, i.e.
a map
\begin{equation*}
    B: (\Omega^\bullet(Z)\otimes\Omega^\bullet(Z))^n\to C^{n-1}(Z;\R)
\end{equation*}
such that
\begin{equation*}
    \omega\wedge \eta - \omega\smile\eta = \delta B(\omega,\eta)+B(d\omega,\eta)+(-1)^{|\omega|}B(\omega,d\eta).
\end{equation*}
It seems that such a homotopy should arise from a proof of the de Rham theorem. A nice
place to look might be the book of Dupont, who proves the de Rham theorem using simplicial
methods. Perhaps Bott and Tu is another good place to look. The product is now defined
\begin{equation*}
    (c_1,h_1,\omega_1)(c_2,h_2,\omega_2) = (c_1\smile c_2,(-1)^{|c_1|}c_1\smile h_2+h_1\smile \omega_2 + B(\omega_1,\omega_2), \omega_1\wedge\omega_2).
\end{equation*}
There are a few things to check: this product descends to a graded-commutative product
on the level of cohomology and that the product is independent of our choice of $B$ (apparently
any two choices are chain homotopic). To check that it descends requires checking that
the product of cocycles is again a cocycle and that the product with a coboundary is zero.
I'll try to type this out tomorrow morning if I'm feeling industrious (and maybe save
the issue about $B$ for later).

\subsubsection{Deligne}
The product for the Deligne model is relatively straightforward, but since I haven't quite
worked through that model yet, I'll postpone the discussion for later.

\section{June 22, 2017}

\subsection{Products in Hopkins-Singer}
I did a few computations this morning in the Hopkins-Singer model, but I ended up
confused. Recall the product of Hopkins-Singer differential cochains from above.
They claim that this product
\begin{equation*}
    \check C(r)^\bullet \otimes \check C(s)^\bullet \to \check C(r+s)^\bullet
\end{equation*}
descends to a graded-commutative product at the level of cohomology
\begin{equation*}
    \check \HH(r)^\bullet \otimes \check \HH(s)^\bullet \to \check\HH(r+s)^\bullet.
\end{equation*}
Let's just check for now that it descends to a product on the differential cohomology
groups. Recall that $\DC^p(M;\Z)=\check\HH(p)^p(M)$ which is the cohomology at
\begin{equation*}
    \begin{tikzcd}
        \check C(p)^{p-1}(M) \rar{d^{p-1}} & \check C(p)^p(M) \rar{d^p} & \check C(p)^{p+1}(M).
    \end{tikzcd}
\end{equation*}
In particular we have that
\begin{align*}
    \check C(p)^p(M) &= C^p(M;\Z) \oplus C^{p-1}(M;\R) \oplus \Omega^p(M)\\
    \check C(p)^{p+1}(M) &= C^{p+1}(M;\Z)\oplus C^p(M;\R)\oplus\Omega^{p+1}(M)\\
    d^p(c,h,\omega) &= (\delta c,c-\omega-\delta h,d\omega)
\end{align*}
and
\begin{align*}
    \check C(p)^{p-1}(M) &= C^{p-1}(M;\Z) \oplus C^{p-2}(M;\R)\\
    d^{p-1}(c,h) &= (\delta c,-c-\delta h,0).
\end{align*}
Suppose we have two cocycles $(c_1,h_1,\omega_1)$ and $(c_2,h_2,\omega_2)$ in
$\check C(p)^p(M)$. Then the product is given
\begin{align*}
    (c_1,h_1,\omega_1)(c_2,h_2,\omega_2) &= (c_1\smile c_2, (-1)^{|c_1|}c_1\smile h_2+h_1\smile\omega_2+B(\omega_1,\omega_2), \omega_1\wedge\omega_2).
\end{align*}
The result should again be a cocycle. Applying $d^{2p}$ we clearly obtain 0 in the first
and third components since $\delta c_1=\delta c _2=0$ and $d\omega_1=d\omega_2=0$.
The formula for $d^{2p}$ (in $\check C(2p)^\bullet(M)$) takes the second component to
\begin{equation*}
    c_1\smile c_2 - \omega_1\wedge\omega_2 - \delta \left( (-1)^{|c_1|}c_1\smile h_2+h_1\smile \omega_2 + B(\omega_1,\omega_2) \right).
\end{equation*}
Recall that
\begin{equation*}
    \delta B(\omega_1,\omega_2) = \omega_1\wedge\omega_2-\omega_1\smile\omega_2-B(d\omega_1,\omega_2)+(-1)^{|\omega_1|}B(\omega_1,d\omega_2)
\end{equation*}
whence we obtain
\begin{equation*}
    c_1\smile c_2 +\omega_1\smile\omega_2-(-1)^{|c_1|}\delta(c_1\smile h_2) - \delta(h_1\smile\omega_2).
\end{equation*}
Since our two cochains are cocycles we have that $\delta h_1=c_1-\omega_1$ and $\delta h_2=c_2-\omega_2$.
Using the Leibniz rule we are now left with
\begin{equation*}
    c_1\smile c_2 + \omega_1\smile\omega_2 - c_1\smile (c_2 -\omega_2) - (c_1 - \omega_1)\smile \omega_2.
\end{equation*}
This is not zero! Instead it simplifies to
\begin{equation*}
    2\omega_1\smile\omega_2 +c_1\smile\omega_2-c_1\smile\omega_2.
\end{equation*}
I run into similar issues when checking that products with coboundaries are again
coboundaries.
\marginnote{What is the problem here?}

\subsection{Deligne cohomology}

Let's look at the smooth Deligne cohomology viewpoint of differential cohomology.
Let $M$ be a smooth manifold.
Define, for $p\geq0$ the $p$th smooth Deligne complex $\Z_{p,\infty}^\bullet$ to be
\begin{equation*}
    \begin{tikzcd}
        \underline{\Z} \rar{\iota} & \mathcal{A}^0 \rar{d} & \mathcal{A}^1 \rar{d} & \cdots \rar{d}& \mathcal{A}^{p-1}
    \end{tikzcd}
\end{equation*}
where $\underline{\Z}$ is the constant sheaf for the group $\Z$ and  
$\mathcal{A}^k$ is the sheaf of $k$-forms on $M$. Recall that we can compute
cohomology of sheaves as right derived functors of the global sections functor. Similarly we
can compute the cohomology of a complex of sheaves. This notion is historically known as
hypercohomology, and goes by the notation $\mathbb{H}$, but I think this is rather silly, so I'll
just refer to it as usual cohomology. The $p$th Deligne cohomology group $\HH_D^p(M;\Z)$ is
defined to be the $p$th sheaf cohomology
\begin{equation*}
    \HH_D^p(M;\Z) = \HH^p(M; \Z_{p,\infty}^\bullet)
\end{equation*}
of the $p$th Deligne complex.

To compute sheaf cohomology recall that we need an acyclic double complex resolution of our
complex. The $p$th cohomology group is then computed by applying global sections to this
resolution and computing the $p$th cohomology of the associated total complex. Let's do
a few simple examples using \v Cech methods. Fix a good open cover $\mathcal{U}$ of $M$
and assume that $M$ is connected.
\marginnote{Why are such \v Cech resolutions acyclic? This should follow from the existence
of partitions of unity.}
Let's first consider the case $p=0$. Then we obtain the complex
$\Z_{0,\infty}^\bullet(M)=\underline{\Z}[0]$, i.e. the complex with just the constant sheaf
$\underline{\Z}$ in degree zero. The 0th Deligne cohomology is thus the 0th sheaf
cohomology of $M$ with coefficients in the constant sheaf $\underline{\Z}$. This is
just the global sections $\Z$. OK, let's now look
at $p=1$. Here we have a two-step complex
$\Z_{1,\infty}^\bullet=\underline{\Z}\to\mathcal{A}^0$.
Taking a \v Cech resolution,
\begin{equation*}
    \begin{tikzcd}
        \mathcal{A}^0 \rar & \underline{C}^0(\mathcal{U},\mathcal{A}^0)\rar{\delta} & \underline{C}^1(\mathcal{U},\mathcal{A}^0)\rar & \cdots\\
        \underline{\Z} \rar\uar{\iota} & \underline{C}^0(\mathcal{U},\underline{\Z}) \rar{\delta}\uar{\iota} & \underline{C}^1(\mathcal{U}, \mathcal{\Z})\rar\uar{\iota} & \cdots
    \end{tikzcd}
\end{equation*}
we drop the first column, calling the result $\check\Z_{1,\infty}^{\bullet,\bullet}$.
To compute the first cohomology of the complex $\Z_{1,\infty}^\bullet$ it suffices to
compute the first cohomology of the global sections functor applied to the total
complex of $\check\Z_{1,\infty}^{\bullet,\bullet}$. This sequence, near degree one, is
\begin{equation*}
    \begin{tikzcd}
        C^0(\mathcal{U},\underline{\Z}) \rar& C^0(\mathcal{U}, \mathcal{A}^0)\oplus C^1(\mathcal{U},\underline{\Z}) \rar & C^1(\mathcal{U},\mathcal{A}^0)\oplus C^2(\mathcal{U}, \underline{\Z})
    \end{tikzcd}
\end{equation*}
where the first map takes $a$ to $(\iota a, \delta a)$
and the second map takes $(f, a)$ to $(\delta f-\iota a, \delta a)$.
A 1-cocycle in this total complex is thus the data of smooth real-valued functions on
each open $U_i$ in the cover which differ on intersections $U_i\cap U_j$ by integers
that satisfy a cocycle condition. A 1-cocycle is a 1-coboundary when the functions
values on the opens $U_i$ are integral. It follows that the first Deligne cohomology
is precisely the group of $\R/\Z$-valued smooth functions on $M$.



\section{June 23, 2017}

\subsection{Deligne cohomology (cont)}
Next let's look at $p=2$. Here we have
$\underline{\Z}_{2,\infty}^\bullet=\underline{\Z}\to\mathcal{A}^0\to\mathcal{A}^1$.
Taking a \v Cech resolution as before we take the total complex and look at the
global sections of the degree 2 term, which looks like
\begin{equation*}
    \begin{tikzcd}
        \check C^0(M;\mathcal{A}^0) \oplus \check C^1(M;\underline{\Z}) \rar &
        \check C^0(M;\mathcal{A}^1)\oplus \check C^1(M;\mathcal{A}^0)\oplus
        \check C^2(M;\underline{\Z})\\
        \;\rar& \check C^1(M;\mathcal{A}^1)\oplus\check C^2(M;\mathcal{A}^0)\oplus
        \check C^3(M;\underline{\Z})
    \end{tikzcd}
\end{equation*}
The first map sends $(f, n)\mapsto (df, \delta f-\iota n, \delta n)$ and
the second map sends $(\alpha, f, n)$ to $(\delta\alpha-df,\delta f+\iota a,\delta a)$.
Hence a coycle is the following data: one-forms $\alpha_i$ on each $U_i$ such that
$\alpha_j-\alpha_i=df_{ij}$, for functions $f_{ij}$ on $U_i\cap U_j$ satisfying
$f_{jk}-f_{ik}+f_{ij}=a_{ijk}$, for $a_{ijk}\in\Z$ satisfying a 3-cocycle condition.
If we exponentiate the conditions on the $f_{ij}$ we obtain nonvanishing complex
functions which serve as transition functions for a complex line bundle. In
particular, we fix trivializing sections $s_i$ on the $U_i$ such that $s_i=e^{2\pi if_{ij}}s_j$
on the overlap. To interpret the one-form data as a connection we write, with
respect to the trivializing section $s_i$ on $U_i$: $\nabla = d + 2\pi i\alpha_i$.
We check that these connection glues to a global connection using the condition on
the $\alpha_i$ above:
\begin{align*}
    \nabla s_i &= (d+2\pi i\alpha_i)(e^{2\pi if_{ij}}s_j) = 2\pi i df_{ij} e^{2\pi if_{ij}} s_j + 2\pi i \alpha_is_j\\
    &= 2\pi i\alpha_j e^{2\pi if_{ij}}s_j = e^{2\pi if_{ij}}\nabla s_j.
\end{align*}
A 2-cocycle is thus the data of a complex line bundle with connection. Finally
let us check that a 2-cocycle yields a trivial line bundle with trivial connection.
A coboundary, the image of say $(g,m)$, is given by the triple $(dg, \delta g-\iota m,\delta m)$.
In particular, the transition functions are given $\exp(2\pi i(g_j-g_i))$
and the local connections are given by multiplication $dg_i$.
Due to the form of the transition functions we can scale the trivialization over each
$U_i$ by $e^{2\pi i g_i}$ to obtain a global trivialization of the line bundle.
\marginnote{What about the connection? Why is it trivial?}

\section{June 26, 2017}

Sean explained to me briefly today how in certain physical models (string theory
of pure spinors or something) the grading on a dga somehow comes after the other
data (in the sense that it can be useful to change it). Sean suggested instead
taking a differential algebra and endowing it with a ``spectral'' vector field $E$, i.e.
a derivation whose eigenelements span the algebra (as an algebra), such that
$E$ commutator the differential is the differential again. This yields a grading
according to the eigenvalues of the generators. We thought this was a kinda funny
definition. The $E$ stands for Euler vector field -- if there is a grading already
then $Ea=|a|a$.

\subsection{Homotopy invariance}

This is a simple formal check.
In \cite{bnv} a presheaf $F$ (on the site of manifolds, with corners I guess) is called
homotopy invariant if for each $M$ the map $F(M)\to F(M\times I)$ induced by the
projection $\pi: M\times I\to M$ is an equivalence. Let's double check that this lines
up with the name -- suppose $M\simeq N$ is a homotopy equivalence of manifolds $M$
and $N$, i.e. we have maps $f:M\to N, g:N\to M$ such that $g\circ f\simeq \id_M$
and $f\circ g\simeq \id_N$.
In particular we have a homotopy $h:M\times I\to M$ between $\id_M$ and $g\circ f$.
We will use the following lemma, which shows that homotopic maps induce equal maps
after applying a homotopy invariant presheaf.
\begin{lemma}
    Let $h:M\times I\to N$ be a homotopy between two maps $f,g:M\to N$.
    Then $Ff=Fg$ if $F$ is homotopy invariant.
\end{lemma}
\begin{proof}
    As above, let $\pi:M\times I\to M$ be the projection and $\iota_0,\iota_1$ be
    the inclusions $M\to M\times I$. Notice that $\pi\circ\iota_1=\pi\circ\iota_0=\id_M$.
    By homotopy invariance of $F$ we know there exists an inverse $(F\pi)^{-1}$.
    Hence $F\iota_0\circ F\pi\circ (F\pi)^{-1}=F\iota_0=(F\pi)^{-1}$ and similarly
    for $F\iota_1$. Thus
    \begin{equation*}
        F\iota_0=F\iota_1=(F\pi)^{-1} : F(M\times I)\to F(M).
    \end{equation*}
    We conclude that $Ff=F\iota_0\circ Fh = F\iota_1\circ Fh=Fg$.
\end{proof}
Applying this to our case we see that $F(g\circ f)=F\id_M=\id_{F(M)}$ and
$F(f\circ g)=F\id_N=\id_{F(N)}$. This shows that a homotopy equivalence induces
isomorphisms after applying a homotopy invariant presheaf.

\subsection{Cochain complexes}

I want to follow Bunke's exposition of the smooth Deligne approach to differential
cohomology. Before we get to the definitions lets recall some basic operations in
the category of cochain complexes (of abelian groups). The first is the shift functor $[-1]$, which 
takes a cochain complex $C^\bullet$ and returns the complex $C^\bullet[-1]$ given
by $C^\bullet[-1] = (C^{p-1}, -d)$. It shifts maps of complexes in the corresponding
manner.  Notice that the shift functor shifts cohomology correspondingly as well.
It is useful to remember the following slogan
(at least given our convention here):
the shift functor shifts the complex to the right and negates the differential.
Of course, the shift has an inverse functor $[1]$ that shifts to the left. Iterated
compositions of these functors will be written $[p]$, $p\in\Z$.
We will write $\sigma$ for the brutal truncation functors. For instance, $\sigma^{\geq p}$
sets all groups in degree lower than $q$ to zero.

Another useful construction we will need is the mapping cone. Let $f:B^\bullet\to C^\bullet$
be a map of cochain complexes. Then we define $\cone(f)$ as
\begin{align*}
    \cone(f)^p &= B^{p+1}\oplus C^p\\
    d(b,c) &= (-db, dc -f(b)).
\end{align*}

\section{June 27, 2017}

\subsection{Cochain complexes (cont)}
Notice that we have a short exact sequence
\begin{equation*}
    \begin{tikzcd}
        0\rar& C\rar& \cone(f)\rar& B[1]\rar& 0
    \end{tikzcd}
\end{equation*}
where the maps are inclusion and minus the projection respectively. The minus seems to
be purely conventional. This short exact sequence yields a long exact sequence in
cohomology
\begin{equation*}
    \begin{tikzcd}
        \HH^{n-1}\cone(f) \rar& \HH^nB\rar& \HH^nC\rar& \HH^n\cone(f)\rar& \HH^{n+1}B
    \end{tikzcd}
\end{equation*}
Our main example will be the following. Considering $\underline{\Z}$ as a complex
of sheaves concentrated in degree zero, consider the direct sum
$\underline{\Z}\oplus\sigma^{\geq n}\Omega$.
Define a map
\begin{equation*}
    \begin{tikzcd}
        \psi:\underline{\Z}\oplus\sigma^{\geq n}\Omega_\C \rar& \Omega_\C
    \end{tikzcd}
\end{equation*}
sending $n\mapsto n\in\Omega_\C$ and $\alpha\mapsto -\alpha\in\Omega_\C$.
The $n$th Deligne complex is defined as a shift to the left of the cone,
\begin{equation*}
    \mathcal{D}(n) = \cone(\psi)[-1],
\end{equation*}
and the $n$th Deligne cohomology is defined as the sheaf (hyper)cohomology
\begin{equation*}
    \Del^n(M;\Z) = \HH^n(M;\mathcal{D}(n)) = \HH^{n-1}(M;\cone\psi).
\end{equation*}

\subsection{The conformal anomaly}
I spent a good part of last quarter studying the determinant line bundle constructions
of Bismut-Freed, but I don't have a good understanding of why they're important.
I only have the rough idea that they are used in making sense of certain path integrals.
There seems to be a somewhat understandable exposition in section 2 of \cite{dts} in
the context of bosonic string theory. These computations are often referred to as the
conformal or Weyl anomaly.

Suppose our goal is to compute the partition function for the Polyakov formulation
of bosonic string theory in a flat Euclidean spacetime $\R^d$.
The partition function is defined naively to be the formal integral
\begin{equation*}
    \mathcal{Z}(X) = \sum_{\text{genus}(\Sigma)=h} \int_{g\in\text{Met}(\Sigma)} [Dg]
        \int_{\phi\in\Map(\Sigma, X)} [d\phi] \exp\left( -\frac{1}{2}\int_\Sigma (d\phi,d\phi)_g \right).
\end{equation*}
The integral over maps $\phi$ can apparently be evaluated in a standard manner,
and yields something. The outer integral, however, is not really what we want --
for whatever reason, we wish to integrate only over the moduli space of genus $h$
surfaces. Hence one has to check that the integrand and the measure descend to
the quotient of the space of metrics by diffeomorphism and conformal group actions.
Apparently the integrand is naturally viewed as a section of a determinant line
bundle over $\text{Met}(\Sigma)$, so the question becomes twofold: does this 
determinant line bundle (together with its metric and connection) descend to the quotient,
and is does the connection trivialize it there? If not, we say that we have an anomaly.
It turns out in this case of the (perturbative) bosonic string there is no anomaly
if our target has dimension $d=26$ (this comes from some computation using the
curvature of the Bismut-Freed connection).

I'd really like to understand the details of what's going on here, but that would
involve understanding the computational details say in the paper of D'Hoker and Phong,
so I'll probably put that off for another day. Perhaps a slightly easier computation
to understand would be the chiral anomaly.


\section{June 28, 2017}

\subsection{Deligne cohomology}
From the definitions above we obtain a long exact sequence in which Deligne cohomology
sits neatly:
the short exact sequence for $\cone\psi$ (for some fixed $n$) is written
\begin{equation*}
    \begin{tikzcd}
        0\rar& \Omega_\C \rar& \cone\psi \rar& (\underline{\Z}\oplus\sigma^{\geq n}\Omega_\C)[1] \rar& 0
    \end{tikzcd}
\end{equation*}
whence we obtain a long exact sequence
\begin{equation*}
    \begin{tikzcd}
        \cdots\rar&\HH^{k-1}(M;\Omega_\C) \rar& \HH^{k-1}(M;\cone\psi) \rar& \HH^k(M;\underline{\Z}\oplus\sigma^{\geq n}\Omega_\C) \rar& \cdots
    \end{tikzcd}
\end{equation*}
We can simplify this using the definition of Deligne cohomology and the natural isomorphism
of various cohomologies. In particular, for $k<n$ we have
\begin{equation*}
    \begin{tikzcd}
        \cdots\rar&\HH^{k-1}_\text{dR}(M;\C) \rar& \HH^k(M;\mathcal{D}(n)) \rar& \HH^k(M;\Z) \rar& \cdots
    \end{tikzcd}
\end{equation*}
for $k=n$ we have,
\begin{equation*}
    \begin{tikzcd}
        \cdots\rar&\HH^{n-1}_\text{dR}(M;\C) \rar& \Del^n(M;\Z) \rar& \HH^n(M;\Z)\oplus\Omega^n_\text{cl}(M;\C) \rar& \cdots
    \end{tikzcd}
\end{equation*}
and for $k>n$ we have
\begin{equation*}
    \begin{tikzcd}
        \cdots\rar&\HH^{k-1}_\text{dR}(M;\C) \rar& \HH^k(M;\mathcal{D}(n)) \rar& \HH^k(M;\Z)\oplus\HH^k_\text{dR}(M;\C) \rar& \cdots
    \end{tikzcd}
\end{equation*}
%Notice that the boundary map $\HH^k(M;\underline{\Z}\oplus\sigma^{\geq n}\Omeag_\C)\to\HH^k(M;\Omega_\C)$
%is just the map induced by $\psi$.

We obtain from this sequence, at $k=n$, the maps
\begin{align*}
    a:& \HH^{n-1}_\text{dR}(M;\C) \to \Del^n(M;\Z)\\
    I:& \Del^n(M;\Z) \to \HH^n(M;\Z)\\
    R:& \Del^n(M;\Z) \to \Omega^n_\text{cl}(M;\C)
\end{align*}
which we will call the ??, underlying class, and curvature maps, respectively.
\marginnote{What is $a$, intuitively? It should measure somehow the failure of homotopy invariance
and/or be related to transgression.}
Indeed these are the sorts of structure maps that arise from any sheaf valued in stable
$\infty$-categories on the \'etale site of manifolds (c.f. \cite{bnv}).


\section{July 7, 2017}

\subsection{Anomalies and determinant lines}

Anomalies are a key part of quantum field theories and are essentially obstructions to the
well-definedness of the path integral (even more so than usual). This obstruction is geometric
in nature and consists of the choice of a unit-norm trivialization of a certain geometric
``determinant line bundle'' associated to the qft.
The mathematical content of this note is a sketch of the construction of this
geometric line bundle, but there will be some physical motivation interspersed.
Indeed, as a motivation for studying families of differential operators, I will first
briefly (and vaguely) mention how determinants appear in path integrals
that involve both bosons and fermions. After defining the determinant line bundle
we will sketch two more concrete examples.
The first is that of the conformal anomaly for bosonic
string theory, which leads to the requirement that spacetime be 26-dimensional.
The second example is the determinant line bundle which arises naturally in the geometric
quantization of Chern-Simons theory, as described by Witten.

\subsubsection{Determinants from path integrals}

Suppose we are interested in a qft with a gauge boson that couples to a fermion (a physical
example is that of qed where we have photons and electrons, respectively) on a spacetime $(X,g)$.
In this case our action can be written as a sum of bosonic and fermionic pieces
\begin{equation*}
    S = S_b + S_f.
\end{equation*}
I will be deliberately vague about where our fields live other than to say that their
configuration spaces are roughly: the bosonic field is an element $(P,A)$ of the the space of
principal $G$-bundles with connection, and the fermionic field $\psi$ is an element of some
associated bundle (with the fibers parity shifted, in the sense of supermanifolds). 
Making this more precise takes a significant amount of work and is not particularly relevant
for us. Neither is the form of $S_b$, but we will suppose that the fermionic action is given
\begin{equation*}
    S_f = \int_X \langle \psi, \fsl D \psi\rangle.
\end{equation*}
Here $\langle-,-\rangle$ is an inner product on the fermionic bundle and the integral
is taken against the volume form (we assume our spacetime is oriented). The operator
$\fsl D$ is a differential operator on the fermionic bundle defined using the connection
$A$. The key point when it comes to the path integral is that the fermionic action
depends on the state of the boson through the covariant derivative $\fsl D$.

The path integral is a functional integral over the space of configurations of the fields
$\phi\in\mathcal{F}$,
\begin{equation*}
    \mathcal{Z} = \int_{\phi\in\mathcal{F}} [d\phi] \; e^{-iS[\phi]/\hbar}.
\end{equation*}
We notice that $\mathcal{F}$ splits as a product $\mathcal{F}=\mathcal{F}_b\times \mathcal{F}_f$
of bosonic and fermionic states whence the integral becomes an iterated integral
\begin{equation*}
    \int_{(P,A)\in \mathcal{F}_b} [d(P,A)]\; e^{-iS[(P,A)]/\hbar} \int_{\psi\in\mathcal{F}_f} [d\psi]\; e^{-i/\hbar\cdot\int_X\langle\psi,\fsl D\psi\rangle}
\end{equation*}
Formally evaluating the fermionic integral, which one treats as a Berezin integral,
one obtains
\begin{equation*}
    \int_{(P,A)\in \mathcal{F}_b} [d(P,A)]\; e^{-iS[(P,A)]/\hbar} \det\fsl D.
\end{equation*}
Recall that this is justified in analogy to the following finite-dimensional argument.
Let $\theta_1,\bar\theta_1,\ldots,\theta_n,\bar\theta_n$ be an oriented real basis of
$\C^n$. Recall that the Berezin integral takes an element of $\Lambda^\bullet\C^n$ and
returns the coefficient of the orientation form
$\theta_1\wedge\bar\theta_1\wedge\cdots\wedge\theta_n\wedge\bar\theta_n$. Let $A$ be
Hermitian matrix, and consider the Berezin integral
\begin{equation*}
    \int \exp -\langle\bar\theta, A\theta\rangle \; d\theta_1\bar d\theta_1\cdots d\theta_n\bar d\theta_n
\end{equation*}
Performing a unitary change of variables we may assume that $A$ is diagonal with
eigenvalues $\lambda_i$. The integral then becomes
\begin{equation*}
    \int (-\sum_{i=1}^n\lambda_i\theta_i\bar\theta_i)^n d\theta_1\bar d\theta_1\cdots d\theta_n\bar d\theta_n
    =\prod_i\lambda_i = \det A.
\end{equation*}
Whatever we mean by $\det\fsl D$, it should certainly vary with the bosonic data $(P,A)$,
and if we wish to compute the remaining path integral, the integrand should be a function.
It turns out that $\det\fsl D$ is naturally viewed as a section of a certain line bundle
over the bosonic fields known as the determinant line bundle. The key question is thus:
is the determinant line bundle trivial? Physicists refer to this problem as the anomaly problem.
If in a theory the determinant $\det\fsl D$ cannot be interpreted as a section of a trivial
line bundle over the space of bosonic fields, we say that the theory has an anomaly. 
When either the determinant line bundle is trivial or the full integrand can be interpreted
as a trivial line bundle (i.e. the bosonic integrand is a section of some bundle whose
topology is dual to that of the determinant line bundle), we say that the theory is anomaly-free
or that we have anomaly cancellation.\footnote{In this heuristic sketch we are completely
ignoring the presence of gauge symmetry.}

The goal of this note is to outline the mathematics behind determinant line bundle.
Let us first note the difficulties in defining $\det\fsl D$. The most
obvious difficulty is that the product of the eigenvalues of $\fsl D$ need not converge,
and so we must regularize in some consistent manner. Slightly more subtle is that in
many cases fermions are sections of $\Z/2$-graded bundles and $\fsl D$ exchanges parity.
In this situation, we have an operator mapping from one vector space to another, so the
notion of determinant or eigenvalue is no longer clear. Finally, and most importantly,
recall that $\fsl D$ is not a single operator but a family of operators parameterized
by the space $\mathcal{F}_b$ of bosonic states. Thus we are forced at the outset
to work in families.

\subsubsection{Families of Dirac operators}

In the following we will deal with the following geometric setup. Let
$\pi:M\to B$ be a family of compact oriented Riemannian $2n$-manifolds. In other
words, $\pi$ is a proper submersion and the vertical tangent bundle $T(M/B)$ is equipped
with an orientation and a metric. Let $E=E^+\oplus E^-\to M$ be a finite-rank complex superbundle
on $M$, equipped with a Hermitian metric $g^E$. We thus obtain a family of vector bundles
$E^z\to M^z$ for each $z\in B$, and a family of Dirac operators is defined to be the data
of smoothly varying odd-parity
differential operators $\d^z:\Gamma(M^z, (E^z)^\pm)\to\Gamma(M^z,(E^z)^\mp)$
such that $(\d^z)^2$ is a generalized Laplacian, i.e. has symbol $\sigma_2(\d^z)^2(x,\xi)=|\xi|^2$.
We will assume that $\d^z$ is formally self-adjoint. Dirac operators are amenable to
analysis due to the existence of the heat kernel (in families), which we will use below.

\begin{exercise}
    Check that $d^z+(d^z)^*$ is a family of formally self-adjoint Dirac operators on the
    exterior algebra of the cotangent bundle of $M$, where the $\Z/2$-grading is given
    by even and odd degree forms.
\end{exercise}

Recall that the determinant of a map $T:V\to V$ of finite dimensional vector spaces is
the induced map $\det T:\det V\to\det V$, where $\det$ is the top exterior power functor.
If our vector space
is $\Z/2$-graded $V=V^+\oplus V^-$ and $T^\pm:V^\pm\to V^\mp$, one adjusts this definition
slightly:
\begin{equation*}
    \det V = (\det V^+)^\vee \otimes \det V^-
\end{equation*}
and we define $\det T^+$ to be the top exterior power of the map $T^+$, which is an element
of $\det V$. Notice that this definition requires that $\dim V^+=\dim V^-$.
These definitions generalize straightforwardly to finite-rank vector bundles. 
If we are to take a similar approach to define $\det\d$, we must first realize $\d$ as
a map of bundles instead of a map of sections. We define the infinite-rank superbundle
$\pi_*E\to B$ to have fibers
\begin{equation*}
    (\pi_*E)_z = \Gamma(M^z, E^z).
\end{equation*}

\begin{exercise}
    Make precise the local triviality of $\pi_*E$.
\end{exercise}

The family $\d$ now provides a map $\d^\pm:\pi_*E^\pm\to\pi_*E^\mp$. Due to the infinite
rank of $\pi_*E$ the procedure described above does not apply here, but let us argue heuristically
for a moment. Suppose we could split $\pi_*E=\oplus H_\lambda$ as finite rank vector bundles
along the spectral subspaces of $\d^2$. Then, since $\d$ commutes with $\d^2$ and $\ker\d=\ker\d^2$
by self-adjointness, $\d^+:H_\lambda^+\to H_\lambda^-$ is an isomorphism for all $\lambda\neq0$.
Formally taking the determinant of the superbundle $\pi_*E$ yields
\begin{equation*}
    \det\pi_*E\cong(\bigotimes_\lambda\det H_\lambda^+)^\vee\otimes\bigotimes_\lambda\det H_\lambda^-
\end{equation*}
Identifying $(\det H_\lambda^+)^\vee\otimes\det H_\lambda^-$ with $\C$ using the isomorphism
$\d^+$ for all $\lambda\neq0$ we are left with
\begin{equation*}
    \det \pi_*E \cong (\det \ker \d^+)^\vee \otimes \det \ker \d^- = \det \ker \d.
\end{equation*}
This heuristic argument leads us to expect that the determinant of $\d$ should be a section of a
line bundle whose fibers are isomorphic to $\det\ker\d^z$.

\subsubsection{The determinant line bundle}

To make the above argument rigorous, we pass to a certain open cover of $B$, construct
determinant line bundles on each open in the cover, and then argue that these bundles glue.
The following result is crucial, and can be proved using heat kernel arguments.
\begin{lemma}
    For $\lambda>0$ let $U_\lambda\subset B$ be the set of $z\in B$ for which $\lambda$ is
    not an eigenvalue of $(\d^z)^2$. Then $\{U_\lambda\}_{\lambda>0}$ is an open cover for
    $B$. Moreover, if $P[0,\lambda):\pi_*E\to\pi_*E$ is the projection onto the spectral subspace
    $[0,\lambda)$ of $\d^2$ then
    \begin{equation*}
        H[0,\lambda) = \im P[0,\lambda)\subset \pi_*E
    \end{equation*}
    is a finite-rank subsuperbundle of $\pi_*E$ over $U_\lambda$.
\end{lemma}
Taking determinants over each $U_\lambda$ we obtain line bundles $\det H[0,\lambda)\to U_\lambda$.
Notice that if $\lambda<\mu$ then over the open set $U_\lambda\cap U_\mu$ we have
$H[0,\mu)=H[0,\lambda) \oplus H(\lambda,\mu)$ whence
\begin{equation*}
    \det H[0,\mu) \cong \det H[0,\lambda) \otimes \det H(\lambda,\mu).
\end{equation*}
The bundle $\det H(\lambda,\mu)$ is a trivial line bundle, as it has a nonvanishing global
section $\det \d^+|_{H(\lambda,\mu)}$. Over $U_\lambda\cap U_\mu$ we thus identify a section
$s$ of $\det H[0,\lambda)$ with $s\otimes \det \d^+|_{H(\lambda,\mu)}$, a section of
$\det H[0,\mu)$. This identification satisfies the cocycle condition
\begin{equation*}
    \det \d^+|_{H(\lambda,\rho)} = \det\d^+|_{H(\mu,\rho)} \otimes \det \d^+|_{H(\lambda,\mu)}
\end{equation*}
on triple overlaps $U_\lambda\cap U_\mu\cap U_\rho$ with $\lambda<\mu<\rho$, so the bundles
$\det H[0,\lambda)$ glue to a global line bundle that we will denote
\begin{equation*}
    \det(\pi_*E,\d^z)\to B.
\end{equation*}
This is the desired determinant line bundle. Notice that its fiber can be identified with
the line that we described heuristically above. We now view the determinant of $\d$ as
a section of the determinant line bundle.
\begin{corollary}
    If $\d$ has index zero there exists a canonical section
    \begin{equation*}
        \det \d^+ \in \Gamma(B, \det(\pi_*E,\d)).
    \end{equation*}
\end{corollary}
\begin{proof}
    The index is defined 
    \begin{equation*}
        \ind \d=\dim\ker\d^+-\dim\ker\d^-.
    \end{equation*}
    When $\ind\d=0$ then (over $U_\lambda$) the map $\d^+|_{H[0,\lambda)}:H[0,\lambda)^+\to H[0,\lambda)^-$
    is a map of vector bundles of the same finite rank. Taking top exterior powers, this map yields
    a section of $\det H[0,\lambda)$ for each $\lambda$. On the overlap $U_\lambda\cap U_\mu$ we have
    $\det\d^+|_{H[0,\mu)}=\det\d^+|_{H[0,\lambda)}\otimes\det\d^+|_{H(\lambda,\mu)}$, so these sections
    glue to a global section of $\det(\pi_*E,\d)$.
\end{proof}
Notice the condition that $\ind\d=0$ is necessary as taking top exterior powers of a linear map is only
possible when the source and target have the same dimension.

We now turn to the question of whether $\det(\pi_*E,\d)\to B$ is trivial as a complex line bundle.
To address this we will introduce some geometry: a Hermitian metric and a compatible connection.
\marginnote{Finding a unit-norm global section.}

\subsubsection{A metric and connection}
The Hermitian metric on $E\to M$ induces a metric on the determinant line bundle known as the
Quillen metric, which we describe now. First we claim that $\pi_*E$ inherits a Hermitian metric
from the metric $(-,-)_E$ on $E$. Indeed, given $s,t\in(\pi_*E)_z=\Gamma(M^z, E^z)$, we define
\begin{equation*}
    \langle s,t\rangle_z = \int_{M^z} (s,t)_E,
\end{equation*}
where the integral is taken against the volume form of $M^z$ induced by the fiberwise Riemannian
metric on $M\to B$. Restricting this metric yields metrics on each $H[0,\lambda)$ and thus
$\det H[0,\lambda)$. We will call this the $L^2$-metric on $U_\lambda$.
Notice, however, that this metric differs from that on $\det H[0,\mu)$
on the overlap $U_\lambda\cap U_\mu$ by the factor
\begin{equation*}
    \lVert \det \d^+|_{H(\lambda,\mu)}\rVert = \prod_{i=1}^m \lambda_i^{1/2},
\end{equation*}
for $\lambda_i$ the eigenvalues of $\d^-\d^+$ lying between $\lambda<\mu$. The dependence on
$z$ is suppressed here for clarity.

Quillen suggested the following trick: scale the $L^2$-metric on $U_\lambda$ by the product
of all (square root of the) eigenvalues of $\d^2$ greater than $\lambda$. In other words,
for $s\in\Gamma(U_\lambda,\det H[0,\lambda))$, define the Quillen metric on $U_\lambda$
\begin{equation*}
    \lVert s\rVert_Q = \lVert s\rVert \cdot \det(P(\lambda,\infty)\d^-\d^+)^{1/2}.
\end{equation*}
If we can regularize the infinite product on the right such that
\begin{equation*}
    \det(P(\lambda,\infty)\d^-\d^+) = \det(P(\mu,\infty)\d^-\d^+) \cdot \prod_{i=1}^m\lambda_i,
\end{equation*}
the Quillen metric on $U_\lambda$ will agree
with the Quillen metric on $U_\mu$ and we will obtain a Hermitian metric on the determinant
line bundle. The regularization can be done using the zeta function. In particular, define
the spectral zeta function
\begin{equation*}
    \zeta(s, \d^-\d^+, \lambda, z) = \frac{1}{\Gamma(s)}\int_0^\infty \tr(P(\lambda,\infty)e^{-t\d^-\d^+})t^{s-1}\;dt.
\end{equation*}
Here $e^{-t\d^-\d^+}:\pi_*E\to\pi_*E$ is the heat kernel for the family $\d^z$ and $s\in\C$
is a complex parameter. It can be shown using the small-time asymptotics of the heat kernel
that the spectral zeta function is meromorphic on $\C$ (the pole set depends in general on
the dimension of the fiber of $M\to B$) but is holomorphic at $s=0$. Moreover, it turns
out that $\zeta$ is smooth in $z$. Hence we define
\begin{equation*}
    \det(P(\lambda,\infty)\d^-\d^+) := \exp(-\zeta'(0,\d^-\d^+,\lambda,z)).
\end{equation*}
\begin{exercise}
    Consider the analogous definition in the case of an automorphism of a finite-dimensional
    vector space and conclude that this is a reasonable definition of determinant.
\end{exercise}
It is a straightforward computation to see that the equation relating $\lambda$ and $\mu$
determinants holds, so we obtain the smooth Quillen metric on $\det(\pi_*E,\d)\to B$ which
locally over $U_\lambda$ is given
\begin{equation*}
    \lVert s\rVert_Q = \lVert s\rVert \cdot \exp(-\zeta'(0,\d^-\d^+,\lambda,z)),
\end{equation*}
where the unmarked norm is the $L^2$-norm on $U_\lambda$ defined above.

Recall that in the case where $\ind\d=0$, we obtained a section $\det\d^+$ of the determinant
line bundle. It follows from the definition of this section that
\begin{equation*}
    \lVert \det\d^+\rVert_Q = \det(\d^-\d^+)^{1/2} = \exp(-\zeta'(0,\d^-\d^+,0,z)).
\end{equation*}

\begin{exercise}
    What is the relation to analytic/Ray-Singer torsion?
\end{exercise}

Now suppose we had a connection $\nabla^E$ on $E\to M$ and a connection on $M\to B$.
Recall that the latter is the data of a splitting $TM=H\oplus T(M/B)$. With this data
we obtain a connection $\nabla^{\pi_*E}$ on $\pi_*E$. Indeed, if
$s\in\Gamma(B,\pi_*E)=\Gamma(M,E)$, given a vector field $X$ on $B$ we can lift it to a
vector field on $M$ using the connection and then apply the covariant derivative $\nabla^E_X$.
One checks that this connection is compatible with the $L^2$-metric on $\pi_*E$.
If we now define
\begin{equation*}
    \nabla^{H[0,\lambda)} = P[0,\lambda)\nabla^{\pi_*E}P[0,\lambda)
\end{equation*}
we obtain induced connections $\nabla^{\det H[0,\lambda)}$ compatible with the induced
$L^2$-metric over $U_\lambda$. Unsurprisingly, these connections do not glue to a connection
on the determinant line bundle: if $s\in\Gamma(U_\lambda\cap U_\mu,\det H[0,\lambda))$ then
\begin{align*}
    \nabla^{\det H[0,\mu)} (s\otimes \det \d^+|_{H(\lambda,\mu)}) &=
        (\nabla^{\det H[0,\lambda)}\otimes \id + \id\otimes\nabla^{\det H(\lambda,\mu)})
            (s\otimes\det\d^+|_{H(\lambda,\mu)})\\
            &= \nabla^{\det H[0,\lambda)}s\otimes\det\d^+|_{H(\lambda,\mu)} + s\otimes\nabla^{\det H(\lambda,\mu)}\det\d^+|_{H(\lambda,\mu)}
\end{align*}
Notice that we obtain an extra term involving the covariant derivative of the
determinant of $\d^+$ when restricted to $H(\lambda,\mu)$.
As we modified the $L^2$-metric above we must modify the connections $\nabla^{\det H[0,\lambda)}$
in order to obtain a global connection. In particular, we might add a one-form
\begin{equation*}
    \beta_\lambda^+ \in \Omega^1(U_\lambda).
\end{equation*}
For the extra term above to disappear we need $\beta_\lambda^+ - \beta_\mu^+=\det\d^+|_{H(\lambda,\mu)}$
over $U_\lambda\cap U_\mu$. The construction of these one-forms is due to Bismut and Freed
whence we obtain a connection on $\det(\pi_*E,\d)$ that over $U_\lambda$ is given
\begin{equation*}
    \nabla^{BF} = \nabla^{\det H[0,\lambda)} + \beta_\lambda^+.
\end{equation*}

For the Bismut-Freed connection to provide a trivialization of the determinant line bundle
it must have zero curvature and trivial holonomy. The curvature is just the square of
the connection. There is a bit of a miracle that occurs here: the one-forms $\beta_\lambda^+$
are obtained
from the degree one components of an inhomogeneous form that appears in the transgression
formula for the local family index theory.\footnote{Actually this is not true
unless we require that the connection $\nabla^E$ be a Clifford connection. Incidentally
this is where we use the even dimension of the fibers $M^z$.}
The details get rather technical so I will not even define $\beta_\lambda^+$ but instead
just assure the reader that there is a beautiful formula for the curvature
\begin{equation*}
    (\nabla^{BF})^2 = (2\pi i)^{-n/2}\left(\int_{M/B}\hat A(M/B)\ch(E/S)\right)_{[2]},
\end{equation*}
where the $[2]$ denotes taking the degree-two differential form component. Here
$\hat A(M/B)$ is the $A$-hat genus of the vertical tangent bundle, $\ch(E/S)$
is the relative Chern character of $E$, and the integral denotes fiber integration.


\subsubsection{The conformal anomaly}

\subsubsection{Chern-Simons theory}


\section{July 25, 2017}

\subsection{Dirac operator in odd dimensions}

Here we follow the exposition in FM05 of the determinant line bundle for the
Dirac operator in odd dimensions. Recall the usual setup (we follow notations from
above as opposed to those of FM05). Let $\pi:M\to B$ be a family
of compact oriented Riemannian manifolds and let $H$ be a horizontal distribution
on this family, i.e. $TM=T(M/B)\oplus H$. Let $E=E^+\oplus E^-\to M$ be a complex Hermitian
superbundle such that $E$ is a selfadjoint module over the Clifford bundle associated to the
relative cotangent bundle $T^*(M/B)$. In other words we have a map
\begin{equation*}
    c: T^*(M/B) \to \End E
\end{equation*}
satisfying
\begin{equation*}
    c(\alpha)c(\beta) + c(\beta)c(\alpha) = - 2 \langle\alpha,\beta\rangle
\end{equation*}
for $\alpha,\beta\in T^*_p(M/B)$ where $p\in M$ and the inner product is the
Hermitian metric on $E$. As is usual in the $\Z/2$-graded case, we ask that
the action be even (even degree elements act as even transformations, etc).
That $E$ is selfadjoint requires (we assume that $E^+$ and $E^-$ are orthogonal)
the operators $c(\alpha)$ be skewadjoint.
\marginnote{Understand the selfadjoint condition better/more explicitly.}
Finally, we let $\nabla^E$ be a Clifford connection on $E$ compatible with the metric.
The Clifford condition requires that
\begin{equation*}
    [\nabla_X^E, c(\alpha)] = c(\nabla^{T^*(M/B)}\alpha)
\end{equation*}
where $\nabla^{T^*(M/B)}$ is the Levi-Civita connection.
\marginnote{Explain where this connection comes from.}
This data yields, for each $b\in B$, the Dirac operator
\begin{equation*}
    \d_b = \gamma \circ \iota_b^*\nabla^E : \Gamma(M_b, E_b) \to \Gamma(M_b, E_b)
\end{equation*}
where $\iota_b: M_b=\pi^{-1}(b)\hookrightarrow M$ is the inclusion. Each Dirac
operator is clearly an odd first-order differential operator. To check that it deserves
to be called a Dirac operator we should moreover check that it squares to a generalized
Laplacian.
\marginnote{Check that it squares to a generalized Laplacian. Mention complexification somewhere.
Selfadjointntess of the Dirac operator.}
One checks that $\d$ is formally selfadjoint.

\begin{example}
    Spinor bundle in both even and odd cases.
    \marginnote{Work through this!}
\end{example}
Now we restrict to the case where our manifolds $M_b$ are odd-dimensional $n=2m+1$. In this case
one checks that multiplication by the relative volume form, an odd endomorphism of $E$,
commutes with Clifford multiplication by any relative cotangent vector. Denote by $\omega$
the product of $i^{m+1}$ with the relative volume form, so that $\omega^2=1$. By definition
of the Dirac operator above, it follows that $c(\omega)$ commutes with $\d$ (in the ungraded
sense). Now the Dirac operator in odd dimensions is defined
\begin{equation*}
    \d'=c(\omega)\circ \d
\end{equation*}
a first-order even selfadjoint operator. Notice that $(\d')^2 = \d^2$ whence $\d'$ has
(by heat kernel arguments) a discrete real spectrum with finite multiplicities.
\marginnote{It looks skewadjoint in half of the odd dimensions..}

\section{July 27, 2017}

I met with Ezra for a bit yesterday and as a result I have the following few goals for the
near future: (1) understand the fractional Pontryagin class, (2) understand the definition
and regularity results of eta invariants, (3) understand the Dirac operator in the paper
of Lukic and Moore in terms of a superconnection. Since the QFT and stable homotopy theory
conference starts next week, I'll be happy if I can do (1) for now.

\begin{proposition}
    Let $p_1\in\HH^4(BSO;\Z)$ be the first Pontryagin class and let $\pi: BSpin\to BSO$
    be the map induced by the double cover. Then there exists a class, which we denote
    by abuse of notation $\frac{p_1}{2}$, such that
    \begin{equation*}
        \pi^*p_1 = 2\cdot \frac{p_1}{2} \in \HH^4(BSpin;\Z).
    \end{equation*}
    We call this the (first) fractional Pontryagin class.
\end{proposition}

To prove this statement we will need a few facts about the homotopy and cohomology groups of
$BSO$ and $BSpin$. First we notice that $Spin$ is 2-connected, whence $BSpin$ is 3-connected.
We know by Bott periodicity that $\pi_3(Spin)\cong\Z$ so Hurewicz implies that
$\pi_4(BSpin)\cong \HH_4(BSpin;\Z)$. Applying the universal coefficient theorem, we find that
$\HH^4(BSpin;\Z)\cong\Z$. This proves the following.

\begin{lemma}
    The first few cohomology groups of $BSpin$ are given
    \begin{align*}
        \HH^0(BSpin;\Z) &\cong \HH^4(BSpin;\Z) \cong \Z\\
        \HH^1(BSpin;\Z) &\cong \HH^2(BSpin;\Z)  \cong \HH^3(BSpin;\Z) \cong 0
    \end{align*}
\end{lemma}

Next recall that $BSO$ has first nontrivial homotopy group $\pi_2(BSO)\cong\Z/2\Z$.
Applying Hurewicz we obtain that $\HH_2(BSO;\Z)\cong\Z/2\Z$.
The universal coefficient theorem shows that $\HH^2(BSO;\Z)\cong0$.

\begin{lemma}
    The first few cohomology groups of $BSO$ are given
    \begin{align*}
        \HH^0(BSO;\Z) & \cong \HH^4(BSO;\Z) \cong \Z\\
        \HH^1(BSO;\Z) & \cong \HH^2 \cong 0\\
        \HH^3(BSO;\Z) & \cong \Z/2\Z
    \end{align*}
\end{lemma}
\begin{proof}
    Use the Bockstein spectral sequence.\marginnote{Work through this.}
\end{proof}

With this data in hand, the proof of the proposition is straightforward (this proof was
suggested to me by Pax Kivimae). We run the Serre spectral sequence for the fibration
\begin{equation*}
    B\Z/2\Z \to BSpin \to BSO
\end{equation*}
associated to the short exact sequence of $0\to\Z/2\Z\to Spin\to SO\to0$ of group. The map
$\pi^*$ arises as an edge homomorphism in degree 4, and the spectral sequence forces
$\pi^*:\HH^4(BSO;\Z)\cong \Z \to \Z\cong \HH^4(BSpin;\Z)$ to be multiplication by two.

\begin{proof}[Proof of proposition]
    <spectral sequence arguments here>
    \marginnote{TeX up the spectral sequences}
    We conclude that the first map in the extension problem
    \begin{equation*}
        0\to \Z \to \HH^4(BSpin;\Z)\cong\Z \to \Z/2\Z \to 0
    \end{equation*}
    is multiplication by two.

    This map is the inclusion $E_\infty^{4,0}\hookrightarrow \HH^4(BSpin;\Z)$, which
    by the naturality of the Serre spectral sequence is precisely the pullback
    $\pi^*:\HH^4(BSO;\Z)\to \HH^4(BSpin;\Z)$.
    \marginnote{Explain the edge morphism}
    %Now we argue that the map $\phi$ is precisely the map $\pi^*$ on $\HH^4(BSO;\Z)$.
    %To see this we use the naturality of the Serre spectral sequence. We consider the
    %map of fibrations
    %\begin{equation*}
    %    \begin{tikzcd}
    %        B\Z/2\Z \rar\dar & BSpin\rar{\pi}\dar{\pi} & BSO\dar\\
    %        * \rar & BSO \rar & BSO
    %    \end{tikzcd}
    %\end{equation*}
    %If we write $E_r$ for the spectral sequence associated to the first row and
    %$\tilde E_r$ for the one associated to the second row, naturality yields maps
    %$\tilde E^{p,q}_r \to E^{p,q}_r$, commuting with differentials. By the above
    %argument we know that $E_\infty^{0,4}\cong\Z/2\Z$ and $E_\infty^{4,0}\cong\Z$.
\end{proof}

\section{August 2, 2017}

\subsection{Large mass limit of a free scalar field theory}

Andy Neitzke, in his lecture on ``Examples of QFTs,'' discussed the (Euclideanized) free scalar field
of mass $m$ and posed the following problem: compute the $m\to\infty$ partition function
for this theory on a closed Riemannian manifold. In particular, one wishes to compute a
suitably regularized quantity
\begin{equation*}
    \mathcal{Z}(M,m) = \frac{1}{\sqrt{\det (\Delta + m^2)}}
\end{equation*}

Sean and I looked at the simple case of the circle, where the operator $\Delta+m^2$ acts on
$L^2(S^1)$. Here $\Delta=-\partial_t^2$ and thus the eigenfunctions are $e^{2\pi int/R}$
with eigenvalues $4\pi^2R^{-2}n^2+m^2$. Let us compute the zeta-regularized determinant of
$\Delta+m^2$. Recall that this is defined
\begin{equation*}
    \det(\Delta+m^2) = \exp(-\zeta'(s=0,\Delta+m^2,\lambda=0)),
\end{equation*}
where
\begin{equation*}
    \zeta(s,H,\lambda) = \frac{1}{\Gamma(s)}\int_0^\infty \tr(P_{(0,\lambda)}e^{-tH}) t^{s-1} dt.
\end{equation*}
Here $e^{-tH}$ is the heat kernel associated to the generalized Laplacian $H$
and $P$ is the spectral projection. It can be proved using heat kernel methods
that $\zeta$ extends meromorphically in $s$ to $\C$ and is regular at $s=0$.

In the case of the circle and $H=\Delta+m^2$ we can explicitly compute the determinant.
Note first that
\begin{equation*}
    \zeta(s,\Delta+m^2,0) = 2\sum_{n=1}^\infty (4\pi^2R^{-2}n^2+m^2)^{-s}.
\end{equation*}
Let us work for $s$ far enough to the right and then analytically continue.
We first differentiate the zeta function with respect to $s$ to obtain
\begin{align*}
    \zeta'(s,\Delta+m^2,0) &= -2\sum_{n=1}^\infty (4\pi^2R^{-2}n^2+m^2)^{-s}
    \cdot \log(4\pi^2R^{-2}n^2+m^2).
\end{align*}
Differentiating in turn with respect to $m$,
\begin{align*}
    \frac{d}{dm}\zeta'(s,\Delta+m^2,0) &= -2\sum_{n=1}^\infty\left(-2ms(4\pi^2R^{-2}n^2+m^2)^{-s-1}
        \cdot\log(4\pi^2R^{-2}n^2+m^2)\right.\\
        & \left. + 2m(4\pi^2R^{-2}n^2+m^2)^{-s-1} \right).
\end{align*}
and taking $s\to0$ we obtain
\marginnote{Be a little more careful here.}
\begin{equation*}
    \frac{d}{dm}\zeta'(s,\Delta+m^2,0) = -\frac{mR^2}{\pi^2}\sum_{n=1}^\infty (n^2+(mR/4\pi)^2)^{-1}.
\end{equation*}
Using the identity
\begin{equation*}
    \sum_{n=1}^\infty (n^2+a^2/\pi^2)^{-1} = \frac{\pi^2}{2}\cdot \frac{a\coth a-1}{a^2}
\end{equation*}
we find
\begin{equation*}
    \frac{d}{dm}\zeta'(s,\Delta+m^2,0) = \frac{2- mR \coth(mR/2)}{m}.
\end{equation*}
Choose a reference mass $m_0>0$ and integrate from $m_0$ to $m>m_0$,
\begin{equation*}
    \zeta'(s,\Delta+m^2,0)-\zeta'(s,\Delta+m_0^2,0) = 2\log \frac{m}{m_0} - 2 \cdot \log\frac{\sinh(mR/2)}{\sinh(m_0R/2)}
\end{equation*}
It follows that
\begin{align*}
    \det(\Delta+m^2) = \frac{\sinh^2(mR/2)/\sinh^2(m_0R/2)}{(m/m_0)^2} \cdot \det(\Delta+m_0^2).
\end{align*}
The partition function as a function of $m$ and $m_0$ is thus computed
\begin{equation*}
    \mathcal{Z}(S^1,m,m_0) = \frac{m/m_0}{\sinh(mR/2)/\sinh(m_0R/2)}  \cdot \mathcal{Z}(S^1,m_0)
\end{equation*}
We conclude that the partition function vanishes in the large mass limit.

A general theme of Andy's lecture was that in a certain limit of renormalization
group flow (here $m\to\infty$) a quantum field theory should become topological.
He suggested that in this particular example of the scalar field theory, the topological
limit should be trivial. I'm not quite sure what ``trivial'' means -- does our
computation here jive with that expectation? At least in the limit it is clear
that the partition function is topological, as we just get 0.

Another interesting case might be $S^2$. The free scalar theory here should be conformal
in the massless limit, so it would be interesting to check whether the zeta-regularization of
$\mathcal{Z}(S^2,m=0)$ is conformal.

\subsection{The $m=0$ partition function}

Suppose $m=0$ in the above discussion. Then the computation is much easier, as we can
just use known values of the zeta function. In this case the zeta function is computed
\begin{equation*}
    \zeta(s,\Delta,0) = 2(2\pi R^{-1})^{-2s}\sum_{n=1}^\infty n^{-2s} = 2(2\pi R^{-1})^{-2s}\zeta(2s).
\end{equation*}
Differentiating, we find
\begin{align*}
    \zeta'(s,\Delta,0) &= -4(2\pi R^{-1})^{-2s}\cdot\log(2\pi R^{-1})\zeta(2s)
    + 4(2\pi R^{-1})^{-2s}\zeta'(2s),
\end{align*}
and taking $s=0$, using
\begin{align*}
    \zeta(0) = -\frac{1}{2} \qquad \zeta'(0) = -\frac{\log 2\pi}{2},
\end{align*}
we obtain
\begin{equation*}
    \zeta'(s=0,\Delta,0) = 2\log(2\pi R^{-1}) - 2 \log 2\pi = -\log R^2.
\end{equation*}
It follows that
\begin{equation*}
    \det \Delta = \exp(-\zeta'(s=0,\Delta,0)) = R^2
\end{equation*}
whence
\begin{equation*}
    \mathcal{Z}(S^1,m=0) = \frac{1}{R}.
\end{equation*}


\section{August 4, 2017}

\subsection{Spherical harmonics}

Let's try a similar calculation as above but one dimension higher. Recall
that the Laplacian on $L^2(S^2)$ has an orthonormal basis consisting of the
spherical harmonics, which have eigenvalue $l(l+1)$ with multiplicity $2l+1$.
The spectral zeta function is thus written
\begin{equation*}
    \zeta(s,\Delta,0) = \sum_{l=1}^\infty (2l+1)l^{-s}(l+1)^{-s}.
\end{equation*}
Let's attempt to compute the zeta-regularized determinant as we did above.
In particular, for large real $s$ we might differentiate,
\begin{equation*}
    \zeta'(s,\Delta,0) = \sum_{l=1}^\infty -(2l+1)\log(l(l+1)) l^{-s}(l+1)^{-s}.
\end{equation*}
We see that we cannot analytically continue to $s=0$, so some other method is
needed. Of course, abstractly (say by heat kernel techniques) we know that
the derivative should be regular at $s=0$. There are a few papers that I should
look into that do these computations.


\subsection{Spin manifolds}

I'd like to understand some explicit examples of spin manifolds and spin connections.
Let's start with the definition.
\begin{definition}
    Let $\pi:E\to X$ be an oriented Riemannian vector bundle of rank $n\geqslant 3$.
    Let $SO(E)\to X$ be the principal bundle of oriented orthonormal frames of $E$.
    Then a \textbf{spin structure} on $E$ is a principal $Spin_n$-bundle $Spin(E)\to X$
    together with a double cover
    \begin{equation*}
        \xi: Spin(E) \to SO(E)
    \end{equation*}
    such that $\xi(pg) = \xi(p)\xi_0(g)$, where $\xi_0:Spin_n\to SO_n$ is the
    double covering map.
\end{definition}

\begin{lemma}
    Spin structures on $E$ are in one-to-one correspondence with 2-sheeted
    coverings of $SO(E)$ that are nontrivial on the fibers of $\pi$.
\end{lemma}
\begin{proof}
    \marginnote{How to lift the group action?}
\end{proof}

\begin{theorem}
    the second stiefel-whitney class vanishes iff spin
\end{theorem}

equivalent definition of spin structure c.f. Sam Gunningham's talk

Levi-Civita connection on tangent bundle, spin connection on both spin structures

\section{August 5, 2017}

\subsection{Spin structure and anomalies}

I'd like to work through parts of Witten's paper ?? in order to understand the following
statement: the theory of the supersymmetric fermion is anomaly-free if and only if there
is a spin structure on spacetime. Or at least that's what my understanding at the moment.

Witten writes the following action for a supersymmetric point particle:
\begin{equation*}
    S = \int d\tau \frac{1}{2}g_{ij}(x(\tau))\frac{dx^i}{d\tau}\frac{dx^j}{d\tau}
    + \frac{i}{2}\psi^i(\tau)\left( g_{ij}\frac{d}{d\tau}+\frac{dx^k}{d\tau}\omega_{kij}(x(\tau)) \right)\psi^j(\tau).
\end{equation*}
Let's try to unpack what's happening here. We have a bosonic field $x$, which is a smooth
map from a connected one-dimensional manifold $T$
\begin{equation*}
    x\in \Map(T, X)
\end{equation*}
where $(X,g)$ is a smooth Riemannian manifold. It has a fermionic superpartner $\psi$,
a section of
\begin{equation*}
    \psi \in \Gamma(A, E=S\otimes x^*\Pi TX).
\end{equation*}
Here $\Pi T^*X$ is the odd tangent bundle of $X$ and $S$ is the complex spinor bundle on
$T$. Indeed, suppose we have chosen a spin structure on $T$, which in particular implies
a metric and an orientation on $T$. Intuitively, we think of the first term being the
kinetic term for the boson. The second term is likely some sort of
$\langle\psi,\fsl D\psi\rangle$ term. There is the Levi-Civita connection on $TX$,
and a spin connection on $S$. This induces a connection $\nabla$ on $E$.
This connection should be (check!) a Clifford connection, whence it determines a
Dirac operator $\fsl D$ on $E$. So more coordinate invariantly, maybe this action can
be written
\begin{equation*}
    S = \frac{1}{2}\int d\tau \lVert \dot x\rVert^2 + i\bar \psi\fsl D\psi.
\end{equation*}

\section{August 14, 2017}

\subsection{More spin basics}

Let $\cl^\times(V,q)$ be the group of units in the Clifford algebra, i.e. the elements
$\phi$ such that $\phi^{-1}\phi=\phi\phi^{-1}=1$. Notice that if $q(v)\neq 0$ then
$v\in\cl^\times(V,q)$, as $v^2=-q(v)$. Notice that this group is a Lie group of dimension
$2^n$ if $n=\dim V$. It's associated Lie algebra is of course the Clifford algebra
equipped with the commutator as the Lie bracket. We define $\Pin(V,q)$ to be the subgroup
of $\cl^\times(V,q)$ generated by $v\in V$ such that $q(v)=\pm 1$. Then the Spin group is
the even degree subgroup of $\Pin(V,q)$,
\begin{equation*}
    \Spin(V,q) = \Pin(V,q) \cap \cl^0(V,q).
\end{equation*}

Let's look at some low-dimensional examples.
\begin{example}
    Let $V=\R$ and $q$ be the usual Euclidean quadratic form. Then $\cl(V,q)\cong\C$.
    Thus $\cl^\times(V,q)=\C^\times$. The only elements of $V$ satisfying $q(v)=\pm 1$
    are, under the identification with $\C$, $\pm i$. Hence $\Pin(V,q)\cong \Z/4$ from
    which it follows that $\Spin(V,q)\cong\Z/2$.
\end{example}

\begin{example}
    Let $V=\R^2$ and $q$ the usual Euclidean quadratic form. Fix an orthonormal basis
    $\{e_1,e_2\}$. Then $e_1^2=e_2^2=-1$ and $e_1e_2=-e_2e_1$. It follows that
    $\cl(V,q)\cong\mathbb{H}$ and $\cl^\times(V,q)=\mathbb{H}^\times$. The unit
    quaternions is just the unit sphere $S^3\subset\R^4$
\end{example}



\bibliographystyle{alpha}

\bibliography{references}

























\end{document}

